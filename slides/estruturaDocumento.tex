\documentclass{beamer}

\usetheme{CambridgeUS}
\usecolortheme{beaver}

\usepackage[utf8]{inputenc}
\usepackage[brazil]{babel}
\usepackage{caption}
\hypersetup{
    final=true,
    % bookmarks=true,
    unicode=false,          % non-Latin characters in Acrobat’s bookmarks
    pdftoolbar=true,        % show Acrobat’s toolbar?
    pdfmenubar=true,        % show Acrobat’s menu?
    pdffitwindow=true,     % window fit to page when opened
    pdfstartview={FitW},    % fits the width of the page to the window
    pdftitle={Estrutura de um documento \LaTeX},    % title
    pdfauthor={Nightwind},     % author
    % pdfsubject={Subject},   % subject of the document
    % pdfcreator={Creator},   % creator of the document
    % pdfproducer={Producer}, % producer of the document
    pdfkeywords={latex, dicas}, % list of keywords
    pdfnewwindow=true,      % links in new PDF window
    colorlinks=false,       % false: boxed links; true: colored links
    linkcolor=red,          % color of internal links (change box color with linkbordercolor)
    citecolor=green,        % color of links to bibliography
    filecolor=cyan,         % color of file links
    urlcolor=magenta        % color of external links
}
\usepackage[style=abnt]{biblatex}
\addbibresource{references.bib}
\usepackage{csquotes}

\usepackage{tabularx}

\usepackage{xcolor}

\usepackage{listings}
\definecolor{mygreen}{rgb}{0,0.6,0}
\definecolor{mygray}{rgb}{0.5,0.5,0.5}
\definecolor{mymauve}{rgb}{0.58,0,0.82}
\definecolor{myback}{rgb}{0.9, 0.9, 0.98}
\definecolor{packagecolor}{rgb}{1.0, 0.0, 0.16}
\definecolor{darkslateblue}{rgb}{0.28, 0.24, 0.55}
\definecolor{forestgreen}{rgb}{0.13, 0.55, 0.13}

\lstdefinestyle{myStyleLatex}{ 
  language=[LaTeX]TeX,
  backgroundcolor = \color{myback},   
  basicstyle = \ttfamily,        
  breakatwhitespace = false,         
  breaklines = true,                
  captionpos = a,                    
  commentstyle = \color{mygreen}, 
  deletekeywords = {...},            
  escapeinside = {{(*}{*)}},        
  extendedchars = true,              
  firstnumber = 1,                
  frame = none,	                  
  keepspaces = true,
  keywordstyle = {\bfseries\color{darkslateblue}},
  keywordstyle = [2]{\bfseries\color{forestgreen}},
  keywordstyle = [3]{\bfseries\color{packagecolor}},
  morekeywords = [2]{options},
  morekeywords = [3]{style,package,document},
  numbers = left,                    
  numbersep = 5pt,                  
  numberstyle = \tiny\color{mygray}, 
  rulecolor = \color{black},         
  showspaces = false,                
  showstringspaces = false,         
  showtabs = false,         
  stepnumber = 1,                   
  stringstyle = \color{mymauve},     
  tabsize = 4,
  title = \lstname                   
}

\usepackage{menukeys}


\title{Estrutura de um documento \LaTeX}
\author{Nightwind}
\institute[CTISM]{Colégio Técnico Industrial de Santa Maria}
\logo{\includegraphics[width=1cm]{../images/photo1.jpg}}
\date{\today}

\begin{document}

\frame{\titlepage}

\begin{frame}
    \frametitle{Sumário}
    \tableofcontents
\end{frame}


\section{Estrutura de ambientes do código}
\begin{frame}[fragile]
    \frametitle{Contexto e introdução}

    \begin{itemize}
        \item O \LaTeX foi desenvolvido para que houvesse maior ``facilidade'' de gerar documentos científicos. Sem uma estruturação bem de comandos bem definida, o código não funciona e os erros se tornam mais frequentes do que o normal.
        \item Para chamar um ``comando'' do \LaTeX, utiliza-se a barra: \lstinline!\!. 
        \item A maioria dos ``comandos'' é seguido de chaves: \lstinline!{}!, o preenchimento delas é obrigatório e varia de acordo com o ``comando''.
        \item Alguns ``comandos'' possuem o compos para preenchimento opcional. Eles são delimitados por colchetes: \lstinline![]!. Eles servem para delimitar opções, configurar o pacote que está sendo chamado, apresentar uma versão reduzida do texto que será impresso (título, seção, instituição).
    \end{itemize}
\end{frame}

\subsection{Formação e função}
\begin{frame}[fragile]
    \frametitle{Formação e função}

    \begin{lstlisting}[style = myStyleLatex]
    \documentclass[options]{style}
    \usepackage[options]{package}
    (*$\vdots$*)
    \title{title}
    \author{names}
    \date{\today}

    \begin{document}
        \maketitle
        (*$\vdots$*)
    \end{document}
    \end{lstlisting}
\end{frame}


\section{Estrutura geral}
\begin{frame}
    \frametitle{Estrutura geral}

    \begin{itemize}
        \item O documento \textit{.tex} é formado por, no mínimo, duas partes: 
        \begin{itemize}
            \item Preâmbulo: onde se faz a declaração dos parâmetros e a conexão com os pacotes.
            \item Documento: onde vai o conteúdo do documento.
        \end{itemize}
    \end{itemize}

\end{frame}
\subsection{Preâmbulo}
\begin{frame}[fragile]
    \frametitle{Preâmbulo}

    \begin{itemize}
        \item Todos os pacotes utilizados precisam ser chamados no preâmbulo.
        \item Ele começa no \lstinline[style=myStyleLatex]!\documentclass[options]{style}! e termina no \lstinline[style=myStyleLatex]!\begin{document}!
    \end{itemize}

\end{frame}
\subsection{Documento}

\section{Estrutura de títulos por classe}
\subsection{article, exam, beamer}
\subsection{book, report}

\section{Referências}
\begin{frame}[allowframebreaks]
    \frametitle{Referências}
    \nocite{WikibooksLatex,BeamerDocumentation,ExamDocumentation,latex24h}
    \printbibliography[]
\end{frame}


\end{document}