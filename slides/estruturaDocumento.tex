\documentclass[brazilian]{beamer}

\usetheme{CambridgeUS}
\usecolortheme{beaver}

% \usepackage[utf8]{inputenc}
\usepackage{babel}
\usepackage{ragged2e}
\apptocmd{\frame}{}{\justifying}{}

\usepackage{xcolor}
\definecolor{emerald}{rgb}{0.31, 0.78, 0.47}

\usepackage{caption}
\usepackage{hyperref}
\hypersetup{
    unicode=true,
    pdfstartview={FitV},
    pdfauthor={Nightwind},
    pdfkeywords={latex, dicas},
    colorlinks=true,
    linkcolor=black,
    citecolor=green,
    filecolor=cyan,
    urlcolor=magenta
}
\usepackage[nameinlink,capitalise]{cleveref}
\usepackage[style=abnt]{biblatex}
\addbibresource{references.bib}
\usepackage{csquotes}

\usepackage{tabularx}
\usepackage{array}
\newcolumntype{M}[1]{>{\raggedleft\arraybackslash}m{#1}}
\newcolumntype{O}[1]{>{\raggedright\arraybackslash}m{#1}}
\newcolumntype{C}[1]{>{\centering\arraybackslash}m{#1}}
\usepackage{multirow}

\usepackage{graphicx}
\usepackage{svg}

\usepackage{multicol}
\usepackage{fontspec}
\setsansfont{Syne}
\setmonofont{Space Mono}
\usefonttheme[onlymath]{serif}

\usepackage{mathtools}
\usepackage{mathrsfs}
\usepackage{xfrac}
\DeclareMathOperator{\sen}{sen}
\newcommand{\intd}[4]{\ensuremath{\int_{#1}^{#2}\left[#3\right]\,\mathsf{d}{#4}}}
\newcommand{\edp}[3][]{\ensuremath{\frac{\partial^{#1}\left[#2\right]}{\partial{#3}^{#1}}}}

\usepackage{listings}
\definecolor{asparagus}{rgb}{0.53, 0.66, 0.42}
\definecolor{lavendergray}{rgb}{0.77, 0.76, 0.82}
\definecolor{pastelmagenta}{rgb}{0.96, 0.6, 0.76}
\definecolor{mistyrose}{rgb}{1.0, 0.89, 0.88}
\definecolor{packagecolor}{rgb}{1.0, 0.0, 0.16}
\definecolor{darkslateblue}{rgb}{0.28, 0.24, 0.55}
\definecolor{forestgreen}{rgb}{0.13, 0.55, 0.13}

\lstdefinestyle{myStyleLatex}{
    language=[LaTeX]TeX,
    backgroundcolor = \color{mistyrose},
    basicstyle = \ttfamily,
    breakatwhitespace = true,
    columns = fullflexible,
    breaklines = true,
    captionpos = a,
    commentstyle = {\footnotesize\color{asparagus}},
    escapeinside = {  {(*}  {*)}  },
    extendedchars = true,
    firstnumber = 1,
    frame = none,
    keepspaces = true,
    keywordstyle = {\bfseries\color{darkslateblue}},
    keywordstyle = [2]{\color{forestgreen}},
    keywordstyle = [3]{\itshape\color{packagecolor}},
    morekeywords = {maketitle,chapter,part, section, subsection, subsubsection, paragraph, subparagraph,familydefault,rmdefault,sfdefault,ttdefault,textsubscript,ttshape,colorbox,textcolor,definecolor,setlength,includegraphics,listoffigures,listoftables,endfirsthead,endhead,endfoot,endlastfoot,arraybackslash,newcolumntype,rowcolor,rowcolors,cellcolor,multirow,setmainfont,documentclass,mathcal,mathbb,mathfrak,mathscr,DeclareMathOperator,cref,autoref,lstdefinestyle,lstset,dfrac,sfrac,lstinline,lstinputlisting},
    morekeywords = [2]{default,arguments},
    morekeywords = [3]{document,command,definition, ambiente,book,report,article,exam, beamer,flushright,flushleft,center,figure,table,tabular,tabularx,longtable,article,book,exam,wrapfigure,equation,split,align,gather,multiline,lstlisting},
    numbers = left,
    numbersep = -1pt,
    numberstyle = \color{lavendergray},
    rulecolor = \color{magenta},
    showspaces = false,
    showstringspaces = false,
    showtabs = false,
    stepnumber = 1,
    stringstyle = \color{pastelmagenta},
    tabsize = 2
}

\usepackage[os=win]{menukeys}

\author{Nightwind}
\institute[CTISM]{Colégio Técnico Industrial de Santa Maria}
\logo{\includegraphics[width=1cm]{../images/photo1.jpg}}
\date{\today}


\title{Estrutura de um documento \LaTeX}


\begin{document}

\frame{\titlepage}

\begin{frame}
    \frametitle{Sumário}
    \tableofcontents
\end{frame}


\section{Estrutura de ambientes do código}
\begin{frame}[fragile]
    \frametitle{Contexto e introdução}
    
    \begin{itemize}
        \item O \LaTeX foi desenvolvido para que houvesse maior ``facilidade'' de gerar documentos científicos. Sem uma estruturação de comandos bem definida, o código não funciona e os erros se tornam mais frequentes do que o normal.
        \item Para chamar um ``comando'' do \LaTeX, utiliza-se a barra: \lstinline!\!.
        \item A maioria dos ``comandos'' é seguido de chaves: \lstinline!{}!, o preenchimento delas é obrigatório e varia de acordo com o ``comando''.
        \item Alguns ``comandos'' possuem o compos para preenchimento opcional. Eles são delimitados por colchetes: \lstinline![]!. Eles servem para escolher opções, configurar o pacote que está sendo chamado, apresentar uma versão reduzida do texto que será impresso (título, seção, instituição).
    \end{itemize}
\end{frame}

\subsection{Formação e função}
\begin{frame}[fragile]
    \frametitle{Formação e função}
    
    \begin{lstlisting}[style = myStyleLatex]
\documentclass[options]{style}
\usepackage[options]{package}
(*$\vdots$*)
\title{title}
\author{names}
\date{\today}

\begin{document}
    \maketitle
    (*$\vdots$*)
\end{document}
    \end{lstlisting}
\end{frame}


\section{Estrutura geral}
\begin{frame}[fragile]
    \frametitle{Estrutura geral}
    
    \begin{itemize}
        \item O documento \textit{.tex} é formado por, no mínimo, duas partes:
              \begin{itemize}
                  \item Preâmbulo: onde se faz a declaração dos parâmetros e a conexão com os pacotes.
                  \item Documento: delimitado por \lstinline[style=myStyleLatex]!\begin{document}\end{document}!, é o ambiente responsável por receber o conteúdo do documento final.
              \end{itemize}
    \end{itemize}
    
\end{frame}
\subsection{Preâmbulo}
\begin{frame}[fragile]
    \frametitle{Preâmbulo}
    
    \begin{itemize}
        \item Todos os pacotes utilizados precisam ser chamados no preâmbulo.
        \item Ele começa no \lstinline[style=myStyleLatex]!\documentclass[options]{style}! e termina no \lstinline[style=myStyleLatex]!\begin{document}!.
        \item É onde se altera o cabeçalho e o rodapé das folhas.
        \item Os comandos \lstinline[style=myStyleLatex]!\renewcommand{command}[arguments][default]{definition}! e \lstinline[style=myStyleLatex]!\newcommand{command}[arguments][default]{definition}! que são listados no preâmbulo, são válidos em todo o documento.
              \begin{itemize}
                  \item Se fosse colocado no início do ambiente, ele mudaria somente aquele ambiente.
                  \item Exemplo: ambiente de listas, tem os contadores ou os ícones alterados através de \lstinline[style=myStyleLatex]!\renewcommand{command}[arguments][default]{definition}!.
              \end{itemize}
    \end{itemize}
\end{frame}
\subsection{Documento}
\begin{frame}[fragile]
    \frametitle{Documento}
    
    \begin{itemize}
        \item É o ambiente que recebe o conteúdo do documento.
        \item Nele constarão todas as divisões em capítulos, seções, subseções, parágrafos, etc.
        \item No geral, para texto corrido não há necessidade de muitos cuidados, ele é muito mais simples de se controlar do que editores de texto comuns.
        \item Alguns destaques que o texto pode receber é com relação ao:
              \begin{itemize}
                  \item negrito: \lstinline[style=myStyleLatex]!\textbf{text}!
                  \item itálico: \lstinline[style=myStyleLatex]!\textit{text}!
              \end{itemize}
    \end{itemize}
\end{frame}

\begin{frame}[fragile]
    \frametitle{Ambientes}
    
    \begin{itemize}
        \item É o ambiente que mais vai receber ``subambientes''. Cada lista, cada equação, cada slide, cada tabela e imagem. Todos estes tem seus próprios ambientes e demandam tratamentos especiais.
        \item Eles são delimitados por:
              \begin{lstlisting}[style=myStyleLatex]
\begin{ambiente}
    (*$\vdots$*)
\end{ambiente}
\end{lstlisting}
    \end{itemize}
    
\end{frame}

\begin{frame}[fragile]
    \frametitle{Títulos}
    
    \begin{itemize}
        \item A estrutura de títulos varia de acordo com a classe:
              \begin{itemize}
                  \item Para \lstinline[style=myStyleLatex]!article, exam, beamer!, hierarquia é: \lstinline[style=myStyleLatex]!part, section, subsection, subsubsection, paragraph, subparagraph!
                  \item Para \lstinline[style=myStyleLatex]!book, report!, há também o comando \lstinline[style=myStyleLatex]!\chapter{text}! antes de \lstinline[style=myStyleLatex]!\section{text}!
              \end{itemize}
    \end{itemize}
    
\end{frame}

\begin{frame}[allowframebreaks]
    \frametitle{Referências}
    \nocite{*}
    \printbibliography[keyword={structure}]
\end{frame}
\end{document}