\documentclass{beamer}

\usetheme{CambridgeUS}
\usecolortheme{beaver}

\usepackage[utf8]{inputenc}
\usepackage[brazil]{babel}
\usepackage{caption}
\hypersetup{
    final=true,
    unicode=false,
    pdftoolbar=true, 
    pdfmenubar=true,
    pdffitwindow=true, 
    pdfstartview={FitW},  
    pdftitle={Inserir Tabelas}, 
    pdfauthor={Nightwind},
    pdfkeywords={latex, dicas}, 
    pdfnewwindow=true,
    colorlinks=false,
    linkcolor=red,
    citecolor=green,
    filecolor=cyan,
    urlcolor=magenta
}
\usepackage[style=abnt]{biblatex}
\addbibresource{references.bib}
\usepackage{csquotes}
\usepackage{cleveref}
\crefname{figure}{Figura}{Figuras}

\usepackage{tabularx}
\usepackage{graphicx}
\usepackage{svg}

\usepackage{xcolor}

\usepackage{listings}
\definecolor{mygreen}{rgb}{0,0.6,0}
\definecolor{mygray}{rgb}{0.5,0.5,0.5}
\definecolor{mymauve}{rgb}{0.58,0,0.82}
\definecolor{myback}{rgb}{0.9, 0.9, 0.98}
\definecolor{packagecolor}{rgb}{1.0, 0.0, 0.16}
\definecolor{darkslateblue}{rgb}{0.28, 0.24, 0.55}
\definecolor{forestgreen}{rgb}{0.13, 0.55, 0.13}

\lstdefinestyle{myStyleLatex}{ 
  language=[LaTeX]TeX,
  backgroundcolor = \color{myback},   
  basicstyle = \ttfamily,        
  breakatwhitespace = false,         
  breaklines = true,                
  captionpos = a,                    
  commentstyle = \color{mygreen}, 
  deletekeywords = {...},            
  escapeinside = {{(*}{*)}},        
  extendedchars = true,              
  firstnumber = 1,                
  frame = none,	                  
  keepspaces = true,
  keywordstyle = {\bfseries\color{darkslateblue}},
  keywordstyle = [2]{\bfseries\color{forestgreen}},
  keywordstyle = [3]{\bfseries\color{packagecolor}},
  morekeywords = [2]{options},
  morekeywords = [3]{style,package,document},
  numbers = left,                    
  numbersep = 5pt,                  
  numberstyle = \tiny\color{mygray}, 
  rulecolor = \color{black},         
  showspaces = false,                
  showstringspaces = false,         
  showtabs = false,         
  stepnumber = 1,                   
  stringstyle = \color{mymauve},     
  tabsize = 4,
  title = \lstname                   
}

\usepackage{menukeys}


\title{Inserir Tabelas}
\author{Nightwind}
\institute[CTISM]{Colégio Técnico Industrial de Santa Maria}
\logo{\includegraphics[width=1cm]{../images/photo1.jpg}}
\date{\today}

\begin{document}

\frame{\titlepage}

\begin{frame}
  \frametitle{Sumário}
  \tableofcontents
\end{frame}

\section{Introdução}
\begin{frame}
  \frametitle{Introdução}

  As tabelas são um ambiente que demanda uma atenção e um cuidado maior. Elas têm uma estrutura bastante rígida e é fácil de se perder quando tem muitas linhas. Por esse motivo, sugiro que utilize sites que gerem o código da tabela, como o \href{https://www.tablesgenerator.com/}{Tables Generator}. Entretanto, em casos de tabelas mais específicas, é válido conhecer pacotes e tratamentos que se adequarão de maneira mais apropriada à situação.

\end{frame}

\section{Tabular}


\section{Tabularx}

\section{Longtable}

\section{Formatação}

\section{Legenda e rótulo}

\section{Lista de tabelas}
\begin{frame}[fragile]
  \frametitle{Lista de tabelas}

  \begin{itemize}
    \item O comando para gerar uma lista de tabelas é \lstinline[style=myStyleLatex]!\listoftables!.
  \end{itemize}

\end{frame}

\section{Referências}
\begin{frame}[allowframebreaks]
  \frametitle{Referências}
  \nocite{*}
  \printbibliography[keyword={inserirTabela}]
\end{frame}


\end{document}