\documentclass[brazilian]{beamer}

\usetheme{CambridgeUS}
\usecolortheme{beaver}

% \usepackage[utf8]{inputenc}
\usepackage{babel}
\usepackage{ragged2e}
\apptocmd{\frame}{}{\justifying}{}

\usepackage{xcolor}
\definecolor{emerald}{rgb}{0.31, 0.78, 0.47}

\usepackage{caption}
\usepackage{hyperref}
\hypersetup{
    unicode=true,
    pdfstartview={FitV},
    pdfauthor={Nightwind},
    pdfkeywords={latex, dicas},
    colorlinks=true,
    linkcolor=black,
    citecolor=green,
    filecolor=cyan,
    urlcolor=magenta
}
\usepackage[nameinlink,capitalise]{cleveref}
\usepackage[style=abnt]{biblatex}
\addbibresource{references.bib}
\usepackage{csquotes}

\usepackage{tabularx}
\usepackage{array}
\newcolumntype{M}[1]{>{\raggedleft\arraybackslash}m{#1}}
\newcolumntype{O}[1]{>{\raggedright\arraybackslash}m{#1}}
\newcolumntype{C}[1]{>{\centering\arraybackslash}m{#1}}
\usepackage{multirow}

\usepackage{graphicx}
\usepackage{svg}

\usepackage{multicol}
\usepackage{fontspec}
\setsansfont{Syne}
\setmonofont{Space Mono}
\usefonttheme[onlymath]{serif}

\usepackage{mathtools}
\usepackage{mathrsfs}
\usepackage{xfrac}
\DeclareMathOperator{\sen}{sen}
\newcommand{\intd}[4]{\ensuremath{\int_{#1}^{#2}\left[#3\right]\,\mathsf{d}{#4}}}
\newcommand{\edp}[3][]{\ensuremath{\frac{\partial^{#1}\left[#2\right]}{\partial{#3}^{#1}}}}

\usepackage{listings}
\definecolor{asparagus}{rgb}{0.53, 0.66, 0.42}
\definecolor{lavendergray}{rgb}{0.77, 0.76, 0.82}
\definecolor{pastelmagenta}{rgb}{0.96, 0.6, 0.76}
\definecolor{mistyrose}{rgb}{1.0, 0.89, 0.88}
\definecolor{packagecolor}{rgb}{1.0, 0.0, 0.16}
\definecolor{darkslateblue}{rgb}{0.28, 0.24, 0.55}
\definecolor{forestgreen}{rgb}{0.13, 0.55, 0.13}

\lstdefinestyle{myStyleLatex}{
    language=[LaTeX]TeX,
    backgroundcolor = \color{mistyrose},
    basicstyle = \ttfamily,
    breakatwhitespace = true,
    columns = fullflexible,
    breaklines = true,
    captionpos = a,
    commentstyle = {\footnotesize\color{asparagus}},
    escapeinside = {  {(*}  {*)}  },
    extendedchars = true,
    firstnumber = 1,
    frame = none,
    keepspaces = true,
    keywordstyle = {\bfseries\color{darkslateblue}},
    keywordstyle = [2]{\color{forestgreen}},
    keywordstyle = [3]{\itshape\color{packagecolor}},
    morekeywords = {maketitle,chapter,part, section, subsection, subsubsection, paragraph, subparagraph,familydefault,rmdefault,sfdefault,ttdefault,textsubscript,ttshape,colorbox,textcolor,definecolor,setlength,includegraphics,listoffigures,listoftables,endfirsthead,endhead,endfoot,endlastfoot,arraybackslash,newcolumntype,rowcolor,rowcolors,cellcolor,multirow,setmainfont,documentclass,mathcal,mathbb,mathfrak,mathscr,DeclareMathOperator,cref,autoref,lstdefinestyle,lstset,dfrac,sfrac,lstinline,lstinputlisting},
    morekeywords = [2]{default,arguments},
    morekeywords = [3]{document,command,definition, ambiente,book,report,article,exam, beamer,flushright,flushleft,center,figure,table,tabular,tabularx,longtable,article,book,exam,wrapfigure,equation,split,align,gather,multiline,lstlisting},
    numbers = left,
    numbersep = -1pt,
    numberstyle = \color{lavendergray},
    rulecolor = \color{magenta},
    showspaces = false,
    showstringspaces = false,
    showtabs = false,
    stepnumber = 1,
    stringstyle = \color{pastelmagenta},
    tabsize = 2
}

\usepackage[os=win]{menukeys}

\author{Nightwind}
\institute[CTISM]{Colégio Técnico Industrial de Santa Maria}
\logo{\includegraphics[width=1cm]{../images/photo1.jpg}}
\date{\today}


\title{Inserir Equações}

\begin{document}
    \frame{\titlepage}
    
    \begin{frame}
        \frametitle{Sumário}
        \tableofcontents
    \end{frame}

\section{Pacotes}
\begin{frame}
    \frametitle{Pacotes}

    Sugestão dos pacotes:
    \begin{itemize}
        \item \texttt{mathtools}: complementa, corrige e substitui o pacote \texttt{amsmath}. 
        \item \texttt{xfrac}: para inserir frações inclinadas.
        \item \texttt{subtack}: para limites com mais de uma linha. 
        \item \texttt{amsfonts}: para ter acesso a mais fontes matemáticas (e.g. Fraktur, Blackboard bold).
        \item \texttt{mathrsfs}: para ter acesso a mais fontes matemáticas (e.g. Script).
    \end{itemize}

\end{frame}


\section{Ambientes}
\begin{frame}[fragile]
    \frametitle{Ambientes}
    Os ambientes de equações no \LaTeX são:

    \begin{table}
        \caption{Ambientes para equações.}
        \label{tab:Ambientes}
        \begin{tabular}{M{4.5cm}cl}
            Comando & Apresentação & Numerada \\ \hline
            \lstinline[style=myStyleLatex]!\(...\)! & corpo de texto & não \\ \hline
            \lstinline[style=myStyleLatex]!$...$! & corpo de texto & não \\ \hline
            \lstinline[style=myStyleLatex]!\[...\]! & em destaque & não \\ \hline
            \lstinline[style=myStyleLatex]!\begin{equation}... \end{equation}! & em destaque & sim \\ \hline
        \end{tabular}
    \end{table}
    Sendo que somente o último pode receber o comando \lstinline[style=myStyleLatex]!\label{<text>}! pois é o único numerado.

\end{frame}


\subsection{Alinhamento}
\begin{frame}[fragile,allowframebreaks]
    \frametitle{Ambientes: alinhamento}

    \begin{itemize}
        \item Existem ambientes específicos para gerarem equações rigidamente alinhadas. Ou para equações que ocupem mais que uma linha. São eles:
        \begin{itemize}
            \item \lstinline[style=myStyleLatex]!\begin{split}...\end{split}!: só pode ser chamado dentro do ambiente \texttt{equation}. Ele possui um nível obrigatório de alinhamento. E infinitas linhas. Toda a equação será numerada uma única vez.
            \item \lstinline[style=myStyleLatex]!\begin{align}...\end{align}!: substitui o ambiente \texttt{equation}. Possui infinitos níveis de alinhamento. Cada linha da equação é numerada individualmente, consequentemente, cada linha pode ser referenciada através do comando \texttt{label}.
            \item \lstinline[style=myStyleLatex]!\begin{gather}...\end{gather}!: substitui o ambiente \texttt{equation}. Centraliza todas as equações citadas no ambiente. Não possui nenhum nível de alinhamento. Cada linha da equação é numerada individualmente, consequentemente, cada linha pode ser referenciada através do comando \texttt{label}.
            \item \lstinline[style=myStyleLatex]!\begin{multiline}...\end{multiline}!: substitui o ambiente \texttt{equation}. Permite a quebra de linha, fazendo com que a equação ocupe duas linhas, mas não dê a impressão que é mais de uma equação. 
        \end{itemize} 
        \item Em todos os ambiente acima descritos (exceto \texttt{split}) é só adicionar um * depois do comando, por exemplo, \texttt{align*}, \texttt{gather*}, \texttt{multiline*}, para que a numeração seja desconsiderada e não exibida. Portanto, não se pode referenciar esse tipo de ambiente porque não tem contador.  
    \end{itemize}

\end{frame}

\section{Símbolos}
\begin{frame}[fragile]
    \frametitle{Símbolos Básicos}

    \begin{table}
        \begin{tabular}{O{1.5cm}l}
            Símbolo & Comando \\ \hline
            \(+\) & \lstinline[style=myStyleLatex]!+! \\ \hline
            \(-\) & \lstinline[style=myStyleLatex]!-! \\ \hline
            \(=\) & \lstinline[style=myStyleLatex]!=! \\ \hline
            \(\times \) & \lstinline[style=myStyleLatex]!\times ! \\ \hline
            \(\div \) & \lstinline[style=myStyleLatex]!\div ! \\ \hline
            \(\neq \) & \lstinline[style=myStyleLatex]!\neq ! \\ \hline
            \(\simeq \) & \lstinline[style=myStyleLatex]!\simeq ! \\ \hline
            \(\pm \) & \lstinline[style=myStyleLatex]!\pm ! \\ \hline
        \end{tabular}
    \end{table}

\end{frame}

\begin{frame}[fragile]
    \frametitle{Símbolos Relacionais}

    \begin{table}
        \begin{tabular}{O{1.5cm}l}
            Símbolo & Comando \\ \hline
            \(<\) & \lstinline[style=myStyleLatex]!<! \\ \hline
            \(>\) & \lstinline[style=myStyleLatex]!>! \\ \hline
            \(\leq\) & \lstinline[style=myStyleLatex]!\leq! \\ \hline
            \(\geq \) & \lstinline[style=myStyleLatex]!\geq! \\ \hline
        \end{tabular}
    \end{table}

\end{frame}

\begin{frame}[fragile]
    \frametitle{Setas}

    \begin{table}
        \begin{tabular}{O{1.5cm}l}
            Símbolo & Comando \\ \hline
            \(\rightarrow  \) & \lstinline[style=myStyleLatex]!\rightarrow ! \\ \hline
            \(\leftarrow \) & \lstinline[style=myStyleLatex]!\leftarrow ! \\ \hline
            \(\Rightarrow \) & \lstinline[style=myStyleLatex]!\Rightarrow ! \\ \hline
            \(\Leftarrow  \) & \lstinline[style=myStyleLatex]!\Leftarrow ! \\ \hline
            \(\longmapsto \) & \lstinline[style=myStyleLatex]!\longmapsto ! \\ \hline
            \(\uparrow \) & \lstinline[style=myStyleLatex]!\uparrow ! \\ \hline
            \(\downarrow \) & \lstinline[style=myStyleLatex]!\downarrow ! \\ \hline
            \(\Uparrow \) & \lstinline[style=myStyleLatex]!\Uparrow ! \\ \hline
            \(\Downarrow \) & \lstinline[style=myStyleLatex]!\Downarrow ! \\ \hline
        \end{tabular}
    \end{table}

\end{frame}

\begin{frame}[fragile]
    \frametitle{Letras gregas}
\tiny
    \begin{table}
        \begin{tabular}{O{1.5cm}l||O{1.5cm}l}
            Símbolo & Comando & Símbolo & Comando \\ \hline
            \(\alpha \) & \lstinline[style=myStyleLatex]!\alpha ! & & \\ \hline
            \(\beta \) & \lstinline[style=myStyleLatex]!\beta ! & & \\ \hline
            \(\gamma \) & \lstinline[style=myStyleLatex]!\gamma ! & \(\Gamma\) & \lstinline[style=myStyleLatex]!\Gamma !\\ \hline
            \(\delta \) & \lstinline[style=myStyleLatex]!\delta ! & \(\Delta \) & \lstinline[style=myStyleLatex]!\Delta !\\ \hline
            \(\epsilon \) & \lstinline[style=myStyleLatex]!\epsilon ! & & \\ \hline
            \(\varepsilon \) & \lstinline[style=myStyleLatex]!\varepsilon ! & & \\ \hline
            \(\zeta \) & \lstinline[style=myStyleLatex]!\zeta ! & & \\ \hline
            \(\eta \) & \lstinline[style=myStyleLatex]!\eta ! & & \\ \hline
            \(\theta \) & \lstinline[style=myStyleLatex]!\theta ! & \(\Theta \) & \lstinline[style=myStyleLatex]!\Theta ! \\ \hline
            \(\iota \) & \lstinline[style=myStyleLatex]!\iota ! & & \\ \hline
            \(\kappa \) & \lstinline[style=myStyleLatex]!\kappa ! & & \\ \hline
            \(\lambda \) & \lstinline[style=myStyleLatex]!\lambda ! & \(\Lambda \) & \lstinline[style=myStyleLatex]!\Lambda ! \\ \hline
            \(\mu \) & \lstinline[style=myStyleLatex]!\mu ! & & \\ \hline
            \(\nu \) & \lstinline[style=myStyleLatex]!\nu ! & & \\ \hline
            \(\xi \) & \lstinline[style=myStyleLatex]!\xi ! & \(\Xi \) & \lstinline[style=myStyleLatex]!\Xi ! \\ \hline
            \(\pi \) & \lstinline[style=myStyleLatex]!\pi ! & \(\Pi \) & \lstinline[style=myStyleLatex]!\Pi ! \\ \hline
            \(\varpi \) & \lstinline[style=myStyleLatex]!\varpi ! & & \\ \hline
            \(\rho \) & \lstinline[style=myStyleLatex]!\rho ! & & \\ \hline
            \(\varrho \) & \lstinline[style=myStyleLatex]!\varrho ! & & \\ \hline
            \(\sigma \) & \lstinline[style=myStyleLatex]!\sigma ! & \(\Sigma \) & \lstinline[style=myStyleLatex]!\Sigma ! \\ \hline
            \(\varsigma \) & \lstinline[style=myStyleLatex]!\varsigma ! & & \\ \hline
            \(\tau \) & \lstinline[style=myStyleLatex]!\tau ! & & \\ \hline
            \(\upsilon \) & \lstinline[style=myStyleLatex]!\upsilon ! & \(\Upsilon \) & \lstinline[style=myStyleLatex]!\Upsilon ! \\ \hline
            \(\phi \) & \lstinline[style=myStyleLatex]!\phi ! & \(\Phi \) & \lstinline[style=myStyleLatex]!\Phi ! \\ \hline
            \(\varphi \) & \lstinline[style=myStyleLatex]!\varphi ! & & \\ \hline
            \(\chi \) & \lstinline[style=myStyleLatex]!\chi ! & & \\ \hline
            \(\psi \) & \lstinline[style=myStyleLatex]!\psi ! & \(\Psi \) & \lstinline[style=myStyleLatex]!\Psi ! \\ \hline
            \(\omega \) & \lstinline[style=myStyleLatex]!\omega ! & \(\Omega \) & \lstinline[style=myStyleLatex]!\Omega ! \\ \hline
        \end{tabular}
    \end{table}

\end{frame}

\begin{frame}[fragile]
    \frametitle{Acentos}

    \begin{table}
        \begin{tabular}{O{1.5cm}l}
            Símbolo & Comando \\ \hline
            \(\hat{a} \) & \lstinline[style=myStyleLatex]!\hat{a} ! \\ \hline
            \(\dot{a} \) & \lstinline[style=myStyleLatex]!\dot{a} ! \\ \hline
            \(\check{a} \) & \lstinline[style=myStyleLatex]!\check{a} ! \\ \hline
            \(\ddot{a} \) & \lstinline[style=myStyleLatex]!\ddot{a} ! \\ \hline
            \(\tilde{a} \) & \lstinline[style=myStyleLatex]!\tilde{a} ! \\ \hline
            \(\breve{a} \) & \lstinline[style=myStyleLatex]!\breve{a} ! \\ \hline
            \(\acute{a} \) & \lstinline[style=myStyleLatex]!\acute{a} ! \\ \hline
            \(\bar{a} \) & \lstinline[style=myStyleLatex]!\bar{a} ! \\ \hline
            \(\grave{a} \) & \lstinline[style=myStyleLatex]!\grave{a} ! \\ \hline
            \(\vec{a} \) & \lstinline[style=myStyleLatex]!\vec{a} ! \\ \hline            
        \end{tabular}
    \end{table}

\end{frame}
\section{Delimitadores}
\begin{frame}[fragile]
    \frametitle{Delimitadores}

    \begin{table}
        \begin{tabular}{O{1.5cm}l}
            Delimitador & Comando \\ \hline
            \(\left\lvert a\right\rvert  \) & \lstinline[style=myStyleLatex]!\left\lvert a\right\rvert ! \\ \hline
            \(\left\lVert a\right\rVert \) & \lstinline[style=myStyleLatex]!\left\lVert a\right\rVert ! \\ \hline
            \(\left(a\right)  \) & \lstinline[style=myStyleLatex]!\left(a\right) ! \\ \hline
            \(\left[a\right] \) & \lstinline[style=myStyleLatex]!\left[a\right] ! \\ \hline
            \(\left\{a\right\} \) & \lstinline[style=myStyleLatex]!\left\{a\right\} ! \\ \hline
            \(\left\langle a\right\rangle \) & \lstinline[style=myStyleLatex]!\left\langle a\right\rangle ! \\ \hline
            \(\left\lfloor a\right\rfloor \) & \lstinline[style=myStyleLatex]!\left\lfloor a\right\rfloor ! \\ \hline
            \(\left\lceil a\right\rceil \) & \lstinline[style=myStyleLatex]!\left\lceil a\right\rceil ! \\ \hline
            \(\left\lfloor a\right\rceil \) & \lstinline[style=myStyleLatex]!\left\lfloor a\right\rceil ! \\ \hline
        \end{tabular}
    \end{table}

\end{frame}

\begin{frame}[fragile]
    \frametitle{Delimitadores}

    \begin{itemize}
        \item Os delimitadores acima expostos são substituíveis por seus equivalentes sem necessitar \lstinline[style=myStyleLatex]!\left...\right! quando o conteúdo interno não for maior que uma linha, por exemplo.
        \item Caso o intuito seja adaptar manualmente o tamanho do delimitador de acordo com o conteúdo é só substituir \lstinline[style=myStyleLatex]!\left...\right! por: \lstinline[style=myStyleLatex]!\big( \Big( \bigg( \Bigg(!.
        \[\big( \Big( \bigg( \Bigg(\]
    \end{itemize}

\end{frame}
\section{Operadores}
\begin{frame}[fragile]
    \frametitle{Operadores}

    \begin{table}
        \begin{tabular}{O{2cm}l}
            Operador & Comando \\ \hline
            \(\arccos \) & \lstinline[style=myStyleLatex]!\arccos ! \\ \hline
            \(\cos \) & \lstinline[style=myStyleLatex]!\cos ! \\ \hline
            \(\arcsin \) & \lstinline[style=myStyleLatex]!\arcsin ! \\ \hline
            \(\sin \) & \lstinline[style=myStyleLatex]!\sin ! \\ \hline
            \(\arctan \) & \lstinline[style=myStyleLatex]!\arctan ! \\ \hline
            \(\tan \) & \lstinline[style=myStyleLatex]!\tan ! \\ \hline
            \(\sec \) & \lstinline[style=myStyleLatex]!\sec ! \\ \hline
            \(\cosh \) & \lstinline[style=myStyleLatex]!\cosh ! \\ \hline
            \(\sinh \) & \lstinline[style=myStyleLatex]!\sinh ! \\ \hline
            \(\lim \) & \lstinline[style=myStyleLatex]!\lim ! \\ \hline
            \(\ln \) & \lstinline[style=myStyleLatex]!\ln ! \\ \hline
            \(\lg \) & \lstinline[style=myStyleLatex]!\lg ! \\ \hline
        \end{tabular}
    \end{table}

\end{frame}
\section{Fontes}
\begin{frame}[fragile]
    \frametitle{Fontes}

    \begin{table}
        \begin{tabular}{rlcc}
            Letra & Comando & Só para & Pacote\\ \hline
            \(\mathcal{AB} \) & \lstinline[style=myStyleLatex]!\mathcal{AB} ! & Maiúsculas & \\ \hline
            \(\mathbb{AB} \) & \lstinline[style=myStyleLatex]!\mathbb{AB} ! & Maiúsculas & \texttt{amssymb} \\ \hline
            \(\mathscr{AB}\) & \lstinline[style=myStyleLatex]!\mathscr{AB}! & Maiúsculas & \texttt{mathrsfs} \\ \hline
            \(\mathfrak{AaBb} \) & \lstinline[style=myStyleLatex]!\mathfrak{AaBb} ! & & \texttt{amssymb}\\ \hline
            \(\mathsf{AaBb} \) & \lstinline[style=myStyleLatex]!\mathsf{AaBb} ! & & \\ \hline
            \(\mathrm{AaBb} \) & \lstinline[style=myStyleLatex]!\mathrm{AaBb} ! & & \\ \hline
            \(\mathbf{AaBb} \) & \lstinline[style=myStyleLatex]!\mathbf{AaBb} ! & & \\ \hline
            \(\mathit{AaBb}\) & \lstinline[style=myStyleLatex]!\mathit{AaBb} ! & & \\ \hline            
        \end{tabular}
    \end{table}

\end{frame}
\section{Novos Comandos}
\begin{frame}[fragile]
    \frametitle{Novos Comandos}

    \begin{itemize}
        \item Determinadas partes nas equações podem se tornar repetitivas e inconvenientes de serem repetidas. 
        \item Por isso, o \LaTeX oferece um meio de tornar a escrita mais simplificada.
        \item \lstinline[style=myStyleLatex]!\newcommand{<cmd>}[<args>][<def>]{<definition>}!
        \item Em que o primeiro campo é preenchido pelo nome do comando, não pode ser repetido.
        \item O segundo campo diz respeito à quantidade de campos o futuro comando irá receber. 
        \item O terceiro campo serve para estabelecer o padrão do \textbf{primeiro} campo preenchível no pelo futuro comando. Se ele for omitido, significa que nenhum campo é opcional. Se ele for preenchido, ocupa o calor no primeiro campo. 
    \end{itemize}

\end{frame}
\begin{frame}[fragile]
    \frametitle{Exemplo}
\footnotesize
\begin{lstlisting}[style=myStyleLatex]
\newcommand{\edp}[3][]{%
    \ensuremath{%
        \frac{\partial^{#1}\left[#2\right]}{\partial{#3}^{#1}}
    }
}
\end{lstlisting}
\lstinline[style=myStyleLatex]!\[\edp[]{3x}{x}\]!
\[\edp[]{3x}{x}\]
\lstinline[style=myStyleLatex]!\[\edp[3]{3y^2}{y}\]!
\[\edp[3]{3y^2}{y}\]

\end{frame}


\section{Novos Operadores}
\begin{frame}[fragile]
    \frametitle{Novos Operadores}

    \begin{itemize}
        \item Para novos operadores, precisamos do pacote \texttt{amsmath}, no mínimo.
    \end{itemize}
\begin{lstlisting}[style=myStyleLatex]
\DeclareMathOperator{<cmd>}{<text>}
\end{lstlisting}

Exemplo: 
\begin{lstlisting}[style=myStyleLatex]
\DeclareMathOperator{\sen}{sen}
\end{lstlisting}

Resultado: \[\sen\theta\]


\end{frame}

\section{Construções Matemáticas}
\subsection{Frações}
\begin{frame}[fragile]
    \frametitle{Frações}
\small
Para fazer uma fração, basta chamar o comando:
\begin{lstlisting}[style=myStyleLatex]
\frac{<numerador>}{<denominador>}
\end{lstlisting}
Exemplo: \[\frac{3}{4}.\]
Nos ambientes que em linhas de texto, a fração vai ficar menor para caber na linha. Exemplo: \(\frac{3}{4}\). Caso o intuito seja deixar ela grande, usar o comando:
\begin{lstlisting}[style=myStyleLatex]
\dfrac{<numerador>}{<denominador>}
\end{lstlisting}
Exemplo: \(\dfrac{3}{4}\).

Para fazer uma fração inclinada, chamar o pacote \texttt{xfrac} e usar o comando:
\begin{lstlisting}[style=myStyleLatex]
\sfrac{<numerador>}{<denominador>}
\end{lstlisting}
Exemplo: \(\sfrac{3}{4}\).

\end{frame}

\subsection{Limites}
\begin{frame}[fragile]
    \frametitle{Limites}

Para o comando de limite, usar:
\begin{lstlisting}[style=myStyleLatex]
\lim_{x \to \infty} 
\end{lstlisting}
Exemplo: \[\lim_{x \to \infty} x \]

\end{frame}

\subsection{Somatórios}

\begin{frame}[fragile]
    \frametitle{Somatórios}

Para o comando de limite, usar:
\begin{lstlisting}[style=myStyleLatex]
\sum_{n = 1}^{\infty} 
\end{lstlisting}
Exemplo: \[\sum_{n = 1}^{\infty} x \]

\end{frame}

\subsection{Raízes}
\begin{frame}[fragile]
    \frametitle{Raízes}

A raiz é um comando formado por um campo opcional e um campo obrigatório. No campo opcional é colocado o índice. E no campo obrigatório é colocado o radicando. 
\begin{lstlisting}[style=myStyleLatex]
\sqrt[<indice>]{<radicando>}
\end{lstlisting}
Exemplo: \[\sqrt{9}\]\[\sqrt[3]{8}\]

\end{frame}

\subsection{Integrais}
\begin{frame}[fragile]
    \frametitle{Integrais}

A integral é representada pelo comando
\begin{lstlisting}[style=myStyleLatex]
\int_{a}^{b} 
\end{lstlisting}
Uma sugestão de complemento seria
\begin{lstlisting}[style=myStyleLatex]
\int_{a}^{b} \left[ x \right] \mathsf{d} x 
\end{lstlisting}
Resultado: \[\int_{a}^{b} \left[ x \right] \mathsf{d} x \]
Pode-se fazer um novo comando para a integral:
\begin{lstlisting}[style=myStyleLatex,]
\newcommand{\intd}[4]{\ensuremath{\int_{#1}^{#2} % 
    \left[ #3 \right] \, \mathsf{d} {#4}}}     
\end{lstlisting}
\[\intd{a}{b}{x}{x}\]
\end{frame}

\subsection{Derivadas}
\begin{frame}[fragile]
    \frametitle{Derivadas}
    Para demonstrar uma derivada, podemos usar:
    \tiny
    \renewcommand{\arraystretch}{3}
\begin{table}[h]
    \begin{tabular}{rl}
        Comando & Resultado \\ \hline
        \lstinline[style=myStyleLatex]! \frac{\mathsf{d}f(t)}{\mathsf{d}t} ! & \(\dfrac{\mathsf{d}f(t)}{\mathsf{d}t}\) \\ \hline
        \lstinline[style=myStyleLatex]! \frac{\mathsf{d}}{\mathsf{d}x}\left[\frac{g(x)}{h(x)}\right] ! & \(\dfrac{\mathsf{d}}{\mathsf{d}x}\left[ \dfrac{g(x)}{h(x)} \right]\) \\ \hline
        \lstinline[style=myStyleLatex]! \frac{\mathsf{d}^4f(t)}{\mathsf{d}t^4} ! & \(\dfrac{\mathsf{d}^4f(t)}{\mathsf{d}t^4}\) \\ \hline
        \lstinline[style=myStyleLatex]! \frac{\partial f(t)}{\partial t} ! & \(\dfrac{\partial f(t)}{\partial t}\) \\ \hline
        \lstinline[style=myStyleLatex]! \frac{\partial}{\partial x}\left[\frac{g(x)}{h(x)}\right] ! & \(\dfrac{\partial}{\partial x}\left[ \dfrac{g(x)}{h(x)} \right]\) \\ \hline
        \lstinline[style=myStyleLatex]! \frac{\partial^4f(t)}{\partial t^4} ! & \(\dfrac{\partial^4f(t)}{\partial t^4}\) \\ \hline
    \end{tabular}
\end{table}

\end{frame}

\section{Rótulo e Referência Cruzada}
\begin{frame}[fragile]
    \frametitle{Rótulo e Referência Cruzada}

    \begin{itemize}
        \item Pode-se referenciar as equações. Para isso, coloca-se um rótulo na equação através do comando \lstinline[style=myStyleLatex]!\label{eq:<text>}!
        \item Para chamar a equação, normalmente precisa compilar duas vezes. 
        \item Usar o comando \lstinline[style=myStyleLatex]!\eqref{eq:>{<text>}}!. 
        \item Com o pacote \texttt{hyperref}, pode se usar o \lstinline[style=myStyleLatex]!\autoref{eq:<text>}!.
        \item Com o pacote \texttt{cleveref}, pode se usar o \lstinline[style=myStyleLatex]!\cref{eq:<text>}!.
    \end{itemize}

\end{frame}


\begin{frame}{allowframebreaks}
    \frametitle{Referências}
    \nocite{*}
    \printbibliography[keyword={inserirEquacoes}]
\end{frame}
\end{document}