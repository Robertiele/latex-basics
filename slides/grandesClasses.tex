\documentclass{beamer}

\usetheme{CambridgeUS}
\usecolortheme{beaver}

\usepackage[utf8]{inputenc}
\usepackage[brazil]{babel}
\usepackage{caption}
\hypersetup{
    final=true,
    % bookmarks=true,
    unicode=false,          % non-Latin characters in Acrobat’s bookmarks
    pdftoolbar=true,        % show Acrobat’s toolbar?
    pdfmenubar=true,        % show Acrobat’s menu?
    pdffitwindow=false,     % window fit to page when opened
    pdfstartview={FitW},    % fits the width of the page to the window
    pdftitle={Grandes classes do \LaTeX},    % title
    pdfauthor={Nightwind},     % author
    %pdfsubject={Subject},   % subject of the document
    % pdfcreator={Creator},   % creator of the document
    % pdfproducer={Producer}, % producer of the document
    pdfkeywords={latex, dicas}, % list of keywords
    pdfnewwindow=true,      % links in new PDF window
    colorlinks=false,       % false: boxed links; true: colored links
    linkcolor=red,          % color of internal links (change box color with linkbordercolor)
    citecolor=green,        % color of links to bibliography
    filecolor=cyan,         % color of file links
    urlcolor=magenta        % color of external links
}
\usepackage[style=abnt]{biblatex}
\addbibresource{references.bib}
\usepackage{csquotes}

\usepackage{tabularx}

\usepackage{xcolor}
\definecolor{mygreen}{rgb}{0,0.6,0}
\definecolor{mygray}{rgb}{0.5,0.5,0.5}
\definecolor{mymauve}{rgb}{0.58,0,0.82}
\usepackage{listings}
\lstset{ 
  backgroundcolor=\color{white},   % choose the background color; you must add \usepackage{color} or \usepackage{xcolor}; should come as last argument
  basicstyle=\ttfamily\footnotesize,        % the size of the fonts that are used for the code
  breakatwhitespace=false,         % sets if automatic breaks should only happen at whitespace
  breaklines=true,                 % sets automatic line breaking
  captionpos=b,                    % sets the caption-position to bottom
  commentstyle=\color{mygreen},    % comment style
  deletekeywords={...},            % if you want to delete keywords from the given language
  escapeinside={\%*}{*)},          % if you want to add LaTeX within your code
  extendedchars=true,              % lets you use non-ASCII characters; for 8-bits encodings only, does not work with UTF-8
  firstnumber=1000,                % start line enumeration with line 1000
  frame=single,	                   % adds a frame around the code
  keepspaces=true,                 % keeps spaces in text, useful for keeping indentation of code (possibly needs columns=flexible)
  keywordstyle=\color{blue},       % keyword style
  language=[LaTeX]TeX,                 % the language of the code
  morekeywords={*,...},            % if you want to add more keywords to the set
  numbers=left,                    % where to put the line-numbers; possible values are (none, left, right)
  numbersep=5pt,                   % how far the line-numbers are from the code
  numberstyle=\tiny\color{mygray}, % the style that is used for the line-numbers
  rulecolor=\color{black},         % if not set, the frame-color may be changed on line-breaks within not-black text (e.g. comments (green here))
  showspaces=false,                % show spaces everywhere adding particular underscores; it overrides 'showstringspaces'
  showstringspaces=false,          % underline spaces within strings only
  showtabs=false,                  % show tabs within strings adding particular underscores
  stepnumber=2,                    % the step between two line-numbers. If it's 1, each line will be numbered
  stringstyle=\color{mymauve},     % string literal style
  tabsize=2,	                   % sets default tabsize to 2 spaces
  title=\lstname                   % show the filename of files included with \lstinputlisting; also try caption instead of title
}
\usepackage{menukeys}

\title{Grandes classes do \LaTeX}
\author{Nightwind}
\institute[CTISM]{Colégio Técnico Industrial de Santa Maria}
\logo{\includegraphics[width=1cm]{../images/photo1.jpg}}
\date{\today}

\begin{document}

\frame{\titlepage}

\begin{frame}
    \frametitle{Sumário}
    \tableofcontents
\end{frame}

\section{Classes}

\begin{frame}[fragile]
    \frametitle{O que são classes no \LaTeX?}

    \begin{itemize}
        \item São modelos de documentos. Elas determinam as grandes configurações, comandos e apresentações do arquivo.
        \item Pode ser que uma classe padrão não apresente o conteúdo da maneira esperada, deste modo, pode-se criar uma classe e carregar ela junto com o documento \lstinline[]!.tex!.
        \item A classe possui o formato \lstinline[]!.cls!.
        \item Sempre que uma classe for criada para uma finalidade, é importante manter junto com o arquivo \lstinline[]!.tex!.
        \item Como escrever uma classe do zero é uma tarefa difícil, há classes padrões que atendem muito bem as necessidades do usuário. 
    \end{itemize}

\end{frame}

\begin{frame}[fragile]
    \frametitle{Classes que podem ser as mais úteis}
\begin{lstlisting}
    \documentclass[<options>]{<class>}
\end{lstlisting}
\begin{tabularx}{\textwidth}{rX}
    \hline
    \textbf{Classe} & \textbf{Características} \\ \hline
    \lstinline[]!article! & Arquivos pequenos e simples, artigos científicos, relatórios mais simples. \\ \hline
    \lstinline[]!beamer! & Padrão para gerar slides, apresentações. \\ \hline
    \lstinline[]!book! & Bastante conteúdo, bastante informação, livros. \\ \hline
    \lstinline[]!exam! & Destinada para a criação de testes, provas, questionários. \\ \hline
    \lstinline[]!report! & Para relatórios longos, com muitos capítulos, seções e etc. \\ \hline
\end{tabularx}
\end{frame}

\section{Opções de Classes}

\begin{frame}
    \frametitle{Opções das Classes}
\begin{table}[h]\small
    \begin{tabularx}{\textwidth}{XX}
        \hline
        \textbf{Opção} & \textbf{Descrição} \\ \hline
        \lstinline[basicstyle={\bfseries\footnotesize}]!10pt!, \lstinline[]!11pt, 12pt! & Define o tamanho padrão da fonte. \\ \hline
        \lstinline[basicstyle={\bfseries\footnotesize}]!letterpaper!, \lstinline[]!a4paper, a5paper, b5paper, executivepaper, legalpaper! & Tamanho do papel. \\ \hline
        \lstinline[]!leqno! & Numeração da equação na margem esquerda. O padrão é que a numeração fique na margem direita. \\ \hline
        \lstinline[]!fleqn! & Alinhar a equação à esquerda. O padrão é que a equação seja centralizada. \\ \hline
        \lstinline!titlepage, notitlepage! & Com ou sem página de título. Em \lstinline!article!, o padrão é sem página de título, em quanto que em \lstinline!book! e \lstinline!report! o padrão é ter uma página para título. \\ \hline
    \end{tabularx}
\end{table}
\end{frame}

\begin{frame}
    \frametitle{Opções das Classes}
\begin{table}[h]\small
    \begin{tabularx}{\textwidth}{XX}
        \hline
        \textbf{Opção} & \textbf{Descrição} \\ \hline
        \lstinline!landscape! & Deixa a página no formato paisagem. \\ \hline
        \lstinline!openright, openany! & Faz com que os capítulos comecem em páginas da direita ou na próxima página disponível. \lstinline!openright! é o padrão para \lstinline!book!. \lstinline!openany! é o padrão para \lstinline!report!. \\ \hline
    \end{tabularx}
\end{table}
\end{frame}

\section{Exemplos}

\begin{frame}[fragile]
    \frametitle{Exemplos}
\begin{lstlisting}
    \documentclass[a4paper,12pt,titlepage]{article}
\end{lstlisting}
\begin{lstlisting}
    \documentclass[a4paper,12pt,openany]{book}
\end{lstlisting}
\begin{lstlisting}
    \documentclass[a4paper,12pt]{exam}
\end{lstlisting}

\end{frame}

\section{Referências}
\begin{frame}
    \frametitle{Referências}
    \nocite{WikibooksLatex,BeamerDocumentation,ExamDocumentation,latex24h}
    \printbibliography[]

\end{frame}


\end{document}
