\documentclass[brazilian]{beamer}
\usetheme{CambridgeUS}
\usecolortheme{beaver}

% \usepackage[utf8]{inputenc}
\usepackage{babel}
\usepackage{ragged2e}
\apptocmd{\frame}{}{\justifying}{}

\usepackage{xcolor}
\definecolor{emerald}{rgb}{0.31, 0.78, 0.47}

\usepackage{caption}
\usepackage{hyperref}
\hypersetup{
    unicode=true,
    pdfstartview={FitV},
    pdfauthor={Nightwind},
    pdfkeywords={latex, dicas},
    colorlinks=true,
    linkcolor=black,
    citecolor=green,
    filecolor=cyan,
    urlcolor=magenta
}
\usepackage[nameinlink,capitalise]{cleveref}
\usepackage[style=abnt]{biblatex}
\addbibresource{references.bib}
\usepackage{csquotes}

\usepackage{tabularx}
\usepackage{array}
\newcolumntype{M}[1]{>{\raggedleft\arraybackslash}m{#1}}
\newcolumntype{O}[1]{>{\raggedright\arraybackslash}m{#1}}
\newcolumntype{C}[1]{>{\centering\arraybackslash}m{#1}}
\usepackage{multirow}

\usepackage{graphicx}
\usepackage{svg}

\usepackage{multicol}
\usepackage{fontspec}
\setsansfont{Syne}
\setmonofont{Space Mono}
\usefonttheme[onlymath]{serif}

\usepackage{mathtools}
\usepackage{mathrsfs}
\usepackage{xfrac}
\DeclareMathOperator{\sen}{sen}
\newcommand{\intd}[4]{\ensuremath{\int_{#1}^{#2}\left[#3\right]\,\mathsf{d}{#4}}}
\newcommand{\edp}[3][]{\ensuremath{\frac{\partial^{#1}\left[#2\right]}{\partial{#3}^{#1}}}}

\usepackage{listings}
\definecolor{asparagus}{rgb}{0.53, 0.66, 0.42}
\definecolor{lavendergray}{rgb}{0.77, 0.76, 0.82}
\definecolor{pastelmagenta}{rgb}{0.96, 0.6, 0.76}
\definecolor{mistyrose}{rgb}{1.0, 0.89, 0.88}
\definecolor{packagecolor}{rgb}{1.0, 0.0, 0.16}
\definecolor{darkslateblue}{rgb}{0.28, 0.24, 0.55}
\definecolor{forestgreen}{rgb}{0.13, 0.55, 0.13}

\lstdefinestyle{myStyleLatex}{
    language=[LaTeX]TeX,
    backgroundcolor = \color{mistyrose},
    basicstyle = \ttfamily,
    breakatwhitespace = true,
    columns = fullflexible,
    breaklines = true,
    captionpos = a,
    commentstyle = {\footnotesize\color{asparagus}},
    escapeinside = {  {(*}  {*)}  },
    extendedchars = true,
    firstnumber = 1,
    frame = none,
    keepspaces = true,
    keywordstyle = {\bfseries\color{darkslateblue}},
    keywordstyle = [2]{\color{forestgreen}},
    keywordstyle = [3]{\itshape\color{packagecolor}},
    morekeywords = {maketitle,chapter,part, section, subsection, subsubsection, paragraph, subparagraph,familydefault,rmdefault,sfdefault,ttdefault,textsubscript,ttshape,colorbox,textcolor,definecolor,setlength,includegraphics,listoffigures,listoftables,endfirsthead,endhead,endfoot,endlastfoot,arraybackslash,newcolumntype,rowcolor,rowcolors,cellcolor,multirow,setmainfont,documentclass,mathcal,mathbb,mathfrak,mathscr,DeclareMathOperator,cref,autoref,lstdefinestyle,lstset,dfrac,sfrac,lstinline,lstinputlisting},
    morekeywords = [2]{default,arguments},
    morekeywords = [3]{document,command,definition, ambiente,book,report,article,exam, beamer,flushright,flushleft,center,figure,table,tabular,tabularx,longtable,article,book,exam,wrapfigure,equation,split,align,gather,multiline,lstlisting},
    numbers = left,
    numbersep = -1pt,
    numberstyle = \color{lavendergray},
    rulecolor = \color{magenta},
    showspaces = false,
    showstringspaces = false,
    showtabs = false,
    stepnumber = 1,
    stringstyle = \color{pastelmagenta},
    tabsize = 2
}

\usepackage[os=win]{menukeys}

\author{Nightwind}
\institute[CTISM]{Colégio Técnico Industrial de Santa Maria}
\logo{\includegraphics[width=1cm]{../images/photo1.jpg}}
\date{\today}


\title{Grandes classes do \LaTeX}

\begin{document}

\frame{\titlepage}

\begin{frame}
    \frametitle{Sumário}
    \tableofcontents
\end{frame}

\section{Classes}

\begin{frame}[fragile]
    \frametitle{O que são classes no \LaTeX?}
    
    \begin{itemize}
        \item São modelos de documentos. Elas determinam as grandes configurações, comandos e apresentações do arquivo.
        \item Pode ser que uma classe padrão não apresente o conteúdo da maneira esperada, deste modo, pode-se criar uma classe e carregar ela junto com o documento \lstinline[style=myStyleLatex]!.tex!.
        \item A classe possui o formato \lstinline[style=myStyleLatex]!.cls!.
        \item Sempre que uma classe for criada para uma finalidade, é importante manter junto com o arquivo \lstinline[style=myStyleLatex]!.tex!.
        \item Como escrever uma classe do zero é uma tarefa difícil, há classes padrões que atendem muito bem as necessidades do usuário.
    \end{itemize}
    
\end{frame}

\begin{frame}[fragile]
    \frametitle{Classes que podem ser as mais úteis}
    \begin{lstlisting}[style=myStyleLatex]
    \documentclass[<options>]{<class>}
\end{lstlisting}
    \begin{tabularx}{\textwidth}{rX}
        \hline
        \textbf{Classe}        & \textbf{Características}                                                   \\ \hline
        \lstinline[style=myStyleLatex]!article! & Arquivos pequenos e simples, artigos científicos, relatórios mais simples. \\ \hline
        \lstinline[style=myStyleLatex]!beamer! & Padrão para gerar slides, apresentações.                                   \\ \hline
        \lstinline[style=myStyleLatex]!book! & Bastante conteúdo, bastante informação, livros.                            \\ \hline
        \lstinline[style=myStyleLatex]!exam! & Destinada para a criação de testes, provas, questionários.                 \\ \hline
        \lstinline[style=myStyleLatex]!report! & Para relatórios longos, com muitos capítulos, seções e etc.                \\ \hline
    \end{tabularx}
\end{frame}

\section{Opções de Classes}

\begin{frame}
    \frametitle{Opções das Classes}
    \begin{table}[h]\small
        \begin{tabularx}{\textwidth}{m{4cm}X}
            \hline
            \textbf{Opção}                                                                                & \textbf{Descrição}                                                                                                                                                         \\ \hline
            \textbf{\texttt{10pt}}, \texttt{11pt, 12pt}                                                   & Define o tamanho padrão da fonte.                                                                                                                                          \\ \hline
            \textbf{\texttt{letterpaper}}, \texttt{a4paper, a5paper, b5paper, executivepaper, legalpaper} & Tamanho do papel.                                                                                                                                                          \\ \hline
            \texttt{leqno}                                                                                & Numeração da equação na margem esquerda. O padrão é que a numeração fique na margem direita.                                                                               \\ \hline
            \texttt{fleqn}                                                                                & Alinhar a equação à esquerda. O padrão é que a equação seja centralizada.                                                                                                  \\ \hline
            \texttt{titlepage, notitlepage}                                                               & Com ou sem página de título. Em \texttt{article}, o padrão é sem página de título, em quanto que em \texttt{book} e \texttt{report} o padrão é ter uma página para título. \\ \hline
        \end{tabularx}
    \end{table}
\end{frame}

\begin{frame}
    \frametitle{Opções das Classes}
    \begin{table}[h]\small
        \begin{tabularx}{\textwidth}{m{4cm}X}
            \hline
            \textbf{Opção}              & \textbf{Descrição}                                                                                                                                                                          \\ \hline
            \texttt{landscape}          & Deixa a página no formato paisagem.                                                                                                                                                         \\ \hline
            \texttt{openright, openany} & Faz com que os capítulos comecem em páginas da direita ou na próxima página disponível. \texttt{openright} é o padrão para \texttt{book}. \texttt{openany} é o padrão para \texttt{report}. \\ \hline
        \end{tabularx}
    \end{table}
\end{frame}

\section{Exemplos}

\begin{frame}[fragile]
    \frametitle{Exemplos}
    \begin{lstlisting}[style=myStyleLatex]
    \documentclass[a4paper,12pt,titlepage]{article}
\end{lstlisting}
    \begin{lstlisting}[style=myStyleLatex]
    \documentclass[a4paper,12pt,openany]{book}
\end{lstlisting}
    \begin{lstlisting}[style=myStyleLatex]
    \documentclass[a4paper,12pt]{exam}
\end{lstlisting}
    
\end{frame}

\begin{frame}
    \frametitle{Referências}
    \nocite{WikibooksLatex,BeamerDocumentation,ExamDocumentation,latex24h}
    \printbibliography[]
    
\end{frame}
\end{document}
