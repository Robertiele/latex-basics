\documentclass{beamer}

\usetheme{CambridgeUS}
\usecolortheme{beaver}

\usepackage[utf8]{inputenc}
\usepackage[brazil]{babel}
\usepackage{caption}
\hypersetup{
    final=true,
    unicode=false,
    pdftoolbar=true, 
    pdfmenubar=true,
    pdffitwindow=true, 
    pdfstartview={FitW},  
    pdftitle={Edição de Texto}, 
    pdfauthor={Nightwind},
    pdfkeywords={latex, dicas}, 
    pdfnewwindow=true,
    colorlinks=false,
    linkcolor=red,
    citecolor=green,
    filecolor=cyan,
    urlcolor=magenta
}
\usepackage[style=abnt]{biblatex}
\addbibresource{references.bib}
\usepackage{csquotes}

\usepackage{tabularx}

\usepackage{xcolor}

\usepackage{listings}
\definecolor{mygreen}{rgb}{0,0.6,0}
\definecolor{mygray}{rgb}{0.5,0.5,0.5}
\definecolor{mymauve}{rgb}{0.58,0,0.82}
\definecolor{myback}{rgb}{0.9, 0.9, 0.98}
\definecolor{packagecolor}{rgb}{1.0, 0.0, 0.16}
\definecolor{darkslateblue}{rgb}{0.28, 0.24, 0.55}
\definecolor{forestgreen}{rgb}{0.13, 0.55, 0.13}

\lstdefinestyle{myStyleLatex}{ 
  language=[LaTeX]TeX,
  backgroundcolor = \color{myback},   
  basicstyle = \ttfamily,        
  breakatwhitespace = false,         
  breaklines = true,                
  captionpos = a,                    
  commentstyle = \color{mygreen}, 
  deletekeywords = {...},            
  escapeinside = {{(*}{*)}},        
  extendedchars = true,              
  firstnumber = 1,                
  frame = none,	                  
  keepspaces = true,
  keywordstyle = {\bfseries\color{darkslateblue}},
  keywordstyle = [2]{\bfseries\color{forestgreen}},
  keywordstyle = [3]{\bfseries\color{packagecolor}},
  morekeywords = [2]{options},
  morekeywords = [3]{style,package,document},
  numbers = left,                    
  numbersep = 5pt,                  
  numberstyle = \tiny\color{mygray}, 
  rulecolor = \color{black},         
  showspaces = false,                
  showstringspaces = false,         
  showtabs = false,         
  stepnumber = 1,                   
  stringstyle = \color{mymauve},     
  tabsize = 4,
  title = \lstname                   
}

\usepackage{menukeys}


\title{Edição de Texto}
\author{Nightwind}
\institute[CTISM]{Colégio Técnico Industrial de Santa Maria}
\logo{\includegraphics[width=1cm]{../images/photo1.jpg}}
\date{\today}

\begin{document}

\frame{\titlepage}

\begin{frame}
    \frametitle{Sumário}
    \tableofcontents
\end{frame}

\section{Fontes}
\begin{frame}[fragile]
    \frametitle{Fontes padrão}

    \begin{itemize}
        \item O \LaTeX tem como padrão a fonte ``Computer Modern''. Essa fonte pode assumir três padrões: \textit{serif}, \textit{sans serif} e \textit{monospaced}.
        \item O padrão \textit{serif} é selecionado automaticamente. 
        \item Para mudar o documento inteiro, utiliza-se \lstinline[style=myStyleLatex]!\renewcommand{\familydefault}{<familia>}!
        \item O campo ``família'' pode assumir: \lstinline[style=myStyleLatex]!\rmdefault, \sfdefault, \ttdefault!
        \item Dentro de um ambiente, pode utilizar os comandos \lstinline[style=myStyleLatex]!\rmfamily, \sffamily, \ttfamily! para alterar o tipo de fonte naquela parte do documento.
    \end{itemize}

\end{frame}

\begin{frame}[fragile]
    \frametitle{Fontes instaladas no PC}

    \begin{itemize}
        \item Utilizar o pacote: \lstinline[style=myStyleLatex]!\usepackage{fontspec}!. 
        \item Avaliar se a fonte garante todas as edições: negrito, itálico, etc.
        \item Para utilizar o pacote, precisa utilizar o LuaLaTeX para compilar.
        \item Se colocar o comando \lstinline[style=myStyleLatex]!\setmainfont{<nome da fonte>}! no preâmbulo para todo o documento. Ou no ambiente que se deseja utilizar.
    \end{itemize}

\end{frame}


\section{Negrito, itálico, sublinhado e riscado}
\begin{frame}[fragile]
    \frametitle{Declaração em bloco}

    \begin{table}[h]
        \begin{tabular}{c|r|l}
            Nome & Comando & Resultado \\ \hline 
            Negrito & \lstinline[style=myStyleLatex]!\textbf{Teste 123}! & \textbf{Teste 123} \\ \hline 
            Normal & \lstinline[style=myStyleLatex]!\textmd{Teste 123}! & \textmd{Teste 123} \\ \hline 
            Itálico & \lstinline[style=myStyleLatex]!\textit{Teste 123}! & \textit{Teste 123} \\ \hline 
            Inclinado & \lstinline[style=myStyleLatex]!\textsl{Teste 123}! & \textsl{Teste 123} \\ \hline
            Sublinhado & \lstinline[style=myStyleLatex]!\underline{Teste 123}! & \underline{Teste 123} \\ \hline 
            Maiúsculas pequenas & \lstinline[style=myStyleLatex]!\textsc{Teste 123}! & \textsc{Teste 123} \\ \hline
            Monoespaçada & \lstinline[style=myStyleLatex]!\texttt{Teste 123}! & \texttt{Teste 123} \\ \hline
            Maiúsculas & \lstinline[style=myStyleLatex]!\uppercase{Teste 123}! & \uppercase{Teste 123} \\ \hline 
            Minúsculas & \lstinline[style=myStyleLatex]!\lowercase{Teste 123}! & \lowercase{Teste 123} \\ \hline 
            Sobreescrito & \lstinline[style=myStyleLatex]!Teste \textsuperscript{123}! & Teste \textsuperscript{123} \\ \hline
            Subscrito & \lstinline[style=myStyleLatex]!Teste \textsubscript{123}! & Teste \textsubscript{123} \\ \hline
        \end{tabular}
    \end{table}

\end{frame}

\begin{frame}[fragile]
    \frametitle{Declaração Global}

    \begin{table}[h]
        \begin{tabular}{c|r}
            Nome & Comando \\ \hline 
            Negrito & \lstinline[style=myStyleLatex]!\bfseries! \\ \hline 
            Normal & \lstinline[style=myStyleLatex]!\mdseries! \\ \hline 
            Itálico & \lstinline[style=myStyleLatex]!\itshape! \\ \hline 
            Inclinado & \lstinline[style=myStyleLatex]!\slshape! \\ \hline
            Maiúsculas pequenas & \lstinline[style=myStyleLatex]!\scshape! \\ \hline
            Monoespaçada & \lstinline[style=myStyleLatex]!\ttshape! \\ \hline
        \end{tabular}
    \end{table}

\end{frame}

\section{Tamanhos de letras}

\begin{frame}[fragile]
    \frametitle{Padrão}

    \begin{itemize}
        \item O tamanho padrão é 10pt.
        \item Outros dois tamanhos podem ser escolhidos nas opções do comando inicial do \LaTeX. 
        \item Pode assumir os valores de 11pt ou 12pt.
    \end{itemize}

\end{frame}

\begin{frame}[fragile]
    \frametitle{Variações de Tamanhos}

    \begin{itemize}
        \item As alterações de tamanho são relativas ao tamanho inicial. 
        \item Se for definição em bloco, utiliza-se o par de chaves. Se for definição global, só declarar o comando (sem chaves) dentro do ambiente.
    \end{itemize}

    \begin{table}[h]
        \begin{tabular}{r|l}
            Comando & Resultado \\ \hline 
            \lstinline[style=myStyleLatex]!\tiny{Teste}! & \tiny{Teste} \\ \hline 
            \lstinline[style=myStyleLatex]!\scriptsize{Teste}! & \scriptsize{Teste} \\ \hline 
            \lstinline[style=myStyleLatex]!\small{Teste}! & \small{Teste} \\ \hline 
            \lstinline[style=myStyleLatex]!\normalsize{Teste}! & \normalsize{Teste} \\ \hline 
            \lstinline[style=myStyleLatex]!\large{Teste}! & \large{Teste} \\ \hline 
            \lstinline[style=myStyleLatex]!\Large{Teste}! & \Large{Teste} \\ \hline 
            \lstinline[style=myStyleLatex]!\LARGE{Teste}! & \LARGE{Teste} \\ \hline
            \lstinline[style=myStyleLatex]!\huge{Teste}! & \huge{Teste} \\ \hline
            \lstinline[style=myStyleLatex]!\Huge{Teste}! & \Huge{Teste} \\ \hline   
        \end{tabular}
    \end{table}

\end{frame}

\begin{frame}[fragile]
    \frametitle{Tamanho específico}

    \begin{itemize}
        \item O comando \lstinline[style=myStyleLatex]!{\fontsize{<size>}{<skip>}\selectfont <texto>}! permite que o usuário determine o valor exato do tamanho da fonte. E o espaçamento para aquele texto.
    \end{itemize}

\end{frame}

\section{Cor de texto e destaque colorido}
\begin{frame}[fragile]
    \frametitle{Informações do Pacote}

    \begin{itemize}
        \item Para utilizar cores, precisamos utilizar um pacote que reconheça cores. Por exemplo: \lstinline[style=myStyleLatex]!\usepackage{xcolor}!
        \item O pacote possui um nome de cores básicas para ser usadas. Outros nomes podem ser obtidos conforme as opções adicionadas na chamada do pacote. Os nomes das cores podem ser encontrados na \href{https://ctan.dcc.uchile.cl/macros/latex/contrib/xcolor/xcolor.pdf}{documentação do pacote}.
        \item Muitas vezes pode ser mais fácil declarar um nova cor baseada no RGB. Um site que dá a numeração pronta: \url{http://latexcolor.com/}.
        \item Exemplo: \lstinline[style=myStyleLatex]!\definecolor{airforceblue}{rgb}{0.36, 0.54, 0.66}!
    \end{itemize}

\end{frame}

\begin{frame}[fragile]
    \frametitle{Texto colorido}

    \begin{itemize}
        \item Pode-se utilizar \lstinline[style=myStyleLatex]!\textcolor{<color>}{<text>}! para que somente um bloco de texto fique colorido.
        \item Para um bloco pequeno de texto com destaque de fundo, utiliza-se \lstinline[style=myStyleLatex]!\colorbox{<color>}{<text>}!.
        \item Se for um texto grande, tem chances de que a linha não quebre e dê problema na hora de compilar, para isso é interessante criar uma caixa de texto: \lstinline[style=myStyleLatex]!\colorbox{<color>}{\parbox{<size box>}{<text>}}!. O tamanho pode ser em: mm, cm, in, pt, ex, em, mu, sp.
    \end{itemize}  

\end{frame}

\section{Alinhamento}
\begin{frame}[fragile]
    \frametitle{Ambiente para alinhamento}

    \begin{itemize}
        \item O alinhamento para o bloco de texto é feito com três ambientes básicos:
        \begin{lstlisting}[style=myStyleLatex]
    \begin{document}
    (*\vdots*)
    \begin{center}
        (*\vdots*)
    \end{center}
    \begin{flushleft}
        (*\vdots*)
    \end{flushleft}
    \begin{flushright}
        (*\vdots*)
    \end{flushright}
    \end{document}  
        \end{lstlisting}
    \end{itemize}

\end{frame}

\begin{frame}[fragile]
    \frametitle{Comando para alinhar}

    \begin{itemize}
        \item Para alinhar seções ou alguns tipos de ambientes, pode-se utilizar os comandos: \lstinline[style=myStyleLatex]!\raggedleft, \raggedright, \centering!.
    \end{itemize}

\end{frame}

\section{Distância entre linhas}
\begin{frame}[fragile]
    \frametitle{Padrão \LaTeX}

    \begin{itemize}
        \item Aplicar o comando no preâmbulo: \lstinline[style=myStyleLatex]!\setlength{\baselineskip}{<size>}! para alterar a distância entre linhas
        \item 
    \end{itemize}

\end{frame}

\section{Recuo de linhas}

\section{Referências}
\begin{frame}[allowframebreaks]
    \frametitle{Referências}
    \nocite{*}
    \printbibliography[keyword={editText}]
\end{frame}


\end{document}