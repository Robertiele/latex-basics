\documentclass{beamer}

\usetheme{CambridgeUS}
\usecolortheme{beaver}

\usepackage[utf8]{inputenc}
\usepackage[brazil]{babel}
\usepackage{caption}
\hypersetup{
    final=true,
    % bookmarks=true,
    unicode=false,          % non-Latin characters in Acrobat’s bookmarks
    pdftoolbar=true,        % show Acrobat’s toolbar?
    pdfmenubar=true,        % show Acrobat’s menu?
    pdffitwindow=true,     % window fit to page when opened
    pdfstartview={FitW},    % fits the width of the page to the window
    pdftitle={Edição de Texto},    % title
    pdfauthor={Nightwind},     % author
    % pdfsubject={Subject},   % subject of the document
    % pdfcreator={Creator},   % creator of the document
    % pdfproducer={Producer}, % producer of the document
    pdfkeywords={latex, dicas}, % list of keywords
    pdfnewwindow=true,      % links in new PDF window
    colorlinks=false,       % false: boxed links; true: colored links
    linkcolor=red,          % color of internal links (change box color with linkbordercolor)
    citecolor=green,        % color of links to bibliography
    filecolor=cyan,         % color of file links
    urlcolor=magenta        % color of external links
}
\usepackage[style=abnt]{biblatex}
\addbibresource{references.bib}
\usepackage{csquotes}

\usepackage{tabularx}

\usepackage{xcolor}

\usepackage{listings}
\definecolor{mygreen}{rgb}{0,0.6,0}
\definecolor{mygray}{rgb}{0.5,0.5,0.5}
\definecolor{mymauve}{rgb}{0.58,0,0.82}
\definecolor{myback}{rgb}{0.9, 0.9, 0.98}
\definecolor{packagecolor}{rgb}{1.0, 0.0, 0.16}
\definecolor{darkslateblue}{rgb}{0.28, 0.24, 0.55}
\definecolor{forestgreen}{rgb}{0.13, 0.55, 0.13}

\lstdefinestyle{myStyleLatex}{ 
  language=[LaTeX]TeX,
  backgroundcolor = \color{myback},   
  basicstyle = \ttfamily,        
  breakatwhitespace = false,         
  breaklines = true,                
  captionpos = a,                    
  commentstyle = \color{mygreen}, 
  deletekeywords = {...},            
  escapeinside = {{(*}{*)}},        
  extendedchars = true,              
  firstnumber = 1,                
  frame = none,	                  
  keepspaces = true,
  keywordstyle = {\bfseries\color{darkslateblue}},
  keywordstyle = [2]{\bfseries\color{forestgreen}},
  keywordstyle = [3]{\bfseries\color{packagecolor}},
  morekeywords = [2]{options},
  morekeywords = [3]{style,package,document},
  numbers = left,                    
  numbersep = 5pt,                  
  numberstyle = \tiny\color{mygray}, 
  rulecolor = \color{black},         
  showspaces = false,                
  showstringspaces = false,         
  showtabs = false,         
  stepnumber = 1,                   
  stringstyle = \color{mymauve},     
  tabsize = 4,
  title = \lstname                   
}

\usepackage{menukeys}


\title{Edição de Texto}
\author{Nightwind}
\institute[CTISM]{Colégio Técnico Industrial de Santa Maria}
\logo{\includegraphics[width=1cm]{../images/photo1.jpg}}
\date{\today}

\begin{document}

\frame{\titlepage}

\begin{frame}
    \frametitle{Sumário}
    \tableofcontents
\end{frame}

\section{Fontes}
\begin{frame}[fragile]
    \frametitle{Fontes padrão}

    \begin{itemize}
        \item O \LaTeX tem como padrão a fonte ``Computer Modern''. Essa fonte pode assumir três padrões: \textit{serif}, \textit{sans serif} e \textit{monospaced}.
        \item O padrão \textit{serif} é selecionado automaticamente. 
        \item Para mudar o documento inteiro, utiliza-se \lstinline[style=myStyleLatex]!\renewcommand{\familydefault}{<familia>}!
        \item O campo ``família'' pode assumir: \lstinline[style=myStyleLatex]!\rmdefault, \sfdefault, \ttdefault!
        \item Dentro de um ambiente, pode utilizar os comandos \lstinline[style=myStyleLatex]!\rmfamily, \sffamily, \ttfamily! para alterar o tipo de fonte naquela parte do documento.
    \end{itemize}

\end{frame}

\begin{frame}[fragile]
    \frametitle{Fontes instaladas no PC}

    \begin{itemize}
        \item Utilizar o pacote: \lstinline[style=myStyleLatex]!\usepackage{fontspec}!. 
        \item Avaliar se a fonte garante todas as edições: negrito, itálico, etc.
        \item Para utilizar o pacote, precisa utilizar o LuaLaTeX para compilar.
        \item Se colocar o comando \lstinline[style=myStyleLatex]!\setmainfont{<nome da fonte>}! no preâmbulo para todo o documento. Ou no ambiente que se deseja utilizar.
    \end{itemize}

\end{frame}


\section{Negrito, itálico, sublinhado e riscado}
\begin{frame}[fragile]
    \frametitle{Declaração em bloco}

    \begin{table}[h]
        \caption{Formatos.}
        \begin{tabular}{c|r|l}
            Nome & Comando & Resultado \\ \hline 
            Negrito & \lstinline[style=myStyleLatex]!\textbf{Teste 123}! & \textbf{Teste 123} \\ \hline 
            Normal & \lstinline[style=myStyleLatex]!\textmd{Teste 123}! & \textmd{Teste 123} \\ \hline 
            Itálico & \lstinline[style=myStyleLatex]!\textit{Teste 123}! & \textit{Teste 123} \\ \hline 
            Inclinado & \lstinline[style=myStyleLatex]!\textsl{Teste 123}! & \textsl{Teste 123} \\ \hline
            Sublinhado & \lstinline[style=myStyleLatex]!\underline{Teste 123}! & \underline{Teste 123} \\ \hline 
            Maiúsculas pequenas & \lstinline[style=myStyleLatex]!\textsc{Teste 123}! & \textsc{Teste 123} \\ \hline
            Monoespaçada & \lstinline[style=myStyleLatex]!\texttt{Teste 123}! & \texttt{Teste 123} \\ \hline
            Maiúsculas & \lstinline[style=myStyleLatex]!\uppercase{Teste 123}! & \uppercase{Teste 123} \\ \hline 
            Minúsculas & \lstinline[style=myStyleLatex]!\lowercase{Teste 123}! & \lowercase{Teste 123} \\ \hline 
            Sobreescrito & \lstinline[style=myStyleLatex]!Teste \textsuperscript{123}! & Teste \textsuperscript{123} \\ \hline
            Subscrito & \lstinline[style=myStyleLatex]!Teste \textsubscript{123}! & Teste \textsubscript{123} \\ \hline
        \end{tabular}
    \end{table}

\end{frame}

\begin{frame}[fragile]
    \frametitle{Declaração Global}

    \begin{table}[h]
        \caption{Formatos.}
        \begin{tabular}{c|r}
            Nome & Comando \\ \hline 
            Negrito & \lstinline[style=myStyleLatex]!\bfseries! \\ \hline 
            Normal & \lstinline[style=myStyleLatex]!\mdseries! \\ \hline 
            Itálico & \lstinline[style=myStyleLatex]!\itshape ! \\ \hline 
            Inclinado & \lstinline[style=myStyleLatex]!\slshape! \\ \hline
            Maiúsculas pequenas & \lstinline[style=myStyleLatex]!\scshape! \\ \hline
            Monoespaçada & \lstinline[style=myStyleLatex]!\ttshape! \\ \hline
        \end{tabular}
    \end{table}

\end{frame}

\section{Tamanhos de letras}

\section{Cor de texto e destaque colorido}

\section{Alinhamento}

\section{Distância entre linhas}

\section{Recuo de linhas}

\section{Referências}
\begin{frame}[allowframebreaks]
    \frametitle{Referências}
    \nocite{*}
    \printbibliography[keyword={editText}]
\end{frame}


\end{document}