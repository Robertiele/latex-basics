\documentclass[brazilian]{beamer}

\usetheme{CambridgeUS}
\usecolortheme{beaver}

% \usepackage[utf8]{inputenc}
\usepackage{babel}
\usepackage{ragged2e}
\apptocmd{\frame}{}{\justifying}{}

\usepackage{xcolor}
\definecolor{emerald}{rgb}{0.31, 0.78, 0.47}

\usepackage{caption}
\usepackage{hyperref}
\hypersetup{
    unicode=true,
    pdfstartview={FitV},
    pdfauthor={Nightwind},
    pdfkeywords={latex, dicas},
    colorlinks=true,
    linkcolor=black,
    citecolor=green,
    filecolor=cyan,
    urlcolor=magenta
}
\usepackage[nameinlink,capitalise]{cleveref}
\usepackage[style=abnt]{biblatex}
\addbibresource{references.bib}
\usepackage{csquotes}

\usepackage{tabularx}
\usepackage{array}
\newcolumntype{M}[1]{>{\raggedleft\arraybackslash}m{#1}}
\newcolumntype{O}[1]{>{\raggedright\arraybackslash}m{#1}}
\newcolumntype{C}[1]{>{\centering\arraybackslash}m{#1}}
\usepackage{multirow}

\usepackage{graphicx}
\usepackage{svg}

\usepackage{multicol}
\usepackage{fontspec}
\setsansfont{Syne}
\setmonofont{Space Mono}
\usefonttheme[onlymath]{serif}

\usepackage{mathtools}
\usepackage{mathrsfs}
\usepackage{xfrac}
\DeclareMathOperator{\sen}{sen}
\newcommand{\intd}[4]{\ensuremath{\int_{#1}^{#2}\left[#3\right]\,\mathsf{d}{#4}}}
\newcommand{\edp}[3][]{\ensuremath{\frac{\partial^{#1}\left[#2\right]}{\partial{#3}^{#1}}}}

\usepackage{listings}
\definecolor{asparagus}{rgb}{0.53, 0.66, 0.42}
\definecolor{lavendergray}{rgb}{0.77, 0.76, 0.82}
\definecolor{pastelmagenta}{rgb}{0.96, 0.6, 0.76}
\definecolor{mistyrose}{rgb}{1.0, 0.89, 0.88}
\definecolor{packagecolor}{rgb}{1.0, 0.0, 0.16}
\definecolor{darkslateblue}{rgb}{0.28, 0.24, 0.55}
\definecolor{forestgreen}{rgb}{0.13, 0.55, 0.13}

\lstdefinestyle{myStyleLatex}{
    language=[LaTeX]TeX,
    backgroundcolor = \color{mistyrose},
    basicstyle = \ttfamily,
    breakatwhitespace = true,
    columns = fullflexible,
    breaklines = true,
    captionpos = a,
    commentstyle = {\footnotesize\color{asparagus}},
    escapeinside = {  {(*}  {*)}  },
    extendedchars = true,
    firstnumber = 1,
    frame = none,
    keepspaces = true,
    keywordstyle = {\bfseries\color{darkslateblue}},
    keywordstyle = [2]{\color{forestgreen}},
    keywordstyle = [3]{\itshape\color{packagecolor}},
    morekeywords = {maketitle,chapter,part, section, subsection, subsubsection, paragraph, subparagraph,familydefault,rmdefault,sfdefault,ttdefault,textsubscript,ttshape,colorbox,textcolor,definecolor,setlength,includegraphics,listoffigures,listoftables,endfirsthead,endhead,endfoot,endlastfoot,arraybackslash,newcolumntype,rowcolor,rowcolors,cellcolor,multirow,setmainfont,documentclass,mathcal,mathbb,mathfrak,mathscr,DeclareMathOperator,cref,autoref,lstdefinestyle,lstset,dfrac,sfrac,lstinline,lstinputlisting},
    morekeywords = [2]{default,arguments},
    morekeywords = [3]{document,command,definition, ambiente,book,report,article,exam, beamer,flushright,flushleft,center,figure,table,tabular,tabularx,longtable,article,book,exam,wrapfigure,equation,split,align,gather,multiline,lstlisting},
    numbers = left,
    numbersep = -1pt,
    numberstyle = \color{lavendergray},
    rulecolor = \color{magenta},
    showspaces = false,
    showstringspaces = false,
    showtabs = false,
    stepnumber = 1,
    stringstyle = \color{pastelmagenta},
    tabsize = 2
}

\usepackage[os=win]{menukeys}

\author{Nightwind}
\institute[CTISM]{Colégio Técnico Industrial de Santa Maria}
\logo{\includegraphics[width=1cm]{../images/photo1.jpg}}
\date{\today}


\title{Inserir Códigos}

\begin{document}

\frame{\titlepage}

\begin{frame}
    \frametitle{Sumário}
    \tableofcontents
\end{frame}
\section{Introdução}

\begin{frame}
    \frametitle{Introdução}
Inserir um código em um documento técnico é uma etapa muito comum, obter sucesso na formatação de espaços e cores adequadas pode ser bem complexo e trabalhoso. 
\end{frame}

\section{Pacote}
\begin{frame}
    \frametitle{Listings}
Para esse tópico, o \LaTeX oferece um ambiente dedicado à inclusão de código em diversas linguagens de programação. E o pacote que usaremos para deixar esse ambiente mais bonito é o \texttt{listings}. Com ele podemos importar o código de um outro arquivo, importar somente uma parte do código, estabelecer o padrão de numeração, editar as cores e o tipo de fonte para os comandos, etc.
\end{frame}

\section{Configuração}
\begin{frame}[fragile]
    \frametitle{Configuração}

    \begin{itemize}
        \item O \texttt{listings} possui dois ``níveis de configuração''.
        \item O nível mais geral é chamado pelo comando \lstinline[style=myStyleLatex]!\lstset{...}!, no qual ocorre a configuração do ambiente de uma maneira geral: fonte, bordas, fundo, número de linhas, etc.
        \item O nível mais específico é chamado de \lstinline[style=myStyleLatex]!\lstdefinestyle{...}!, em que consta as configurações específicas da linguagem.
        \item Se haverá somente uma linguagem de programação, não há necessidade de configurar o estilo.
    \end{itemize}

\end{frame}

\begin{frame}[fragile]
    \frametitle{Configuração}
\tiny
    \begin{table}
        \begin{tabular}{rcl}
            Opção & Preenchimento & Explicação \\ \hline
            \texttt{backgroundcolor} & \lstinline[style=myStyleLatex]!\color{name}! & Cor de fundo para o ambiente \\ \hline
            \texttt{basicstyle} & - & É o padrão da escrita básica vai obedecer: tamanho, fonte, etc.  \\ \hline
            \texttt{u} & \texttt{\bfseries u} \texttt{7} & gg \\ \hline
        \end{tabular}
    \end{table}

\end{frame}

\section{Inclusão de código}

\section{Inclusão de documento}

\section{Legendas e Rótulos}

\begin{frame}[allowframebreaks]
    \frametitle{Referências}

    \nocite{*}
    \printbibliography[keyword={inserirCodigos}]

\end{frame}

\end{document}