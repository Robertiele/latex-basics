\documentclass[brazilian]{beamer}

\usetheme{CambridgeUS}
\usecolortheme{beaver}

% \usepackage[utf8]{inputenc}
\usepackage{babel}
\usepackage{ragged2e}
\apptocmd{\frame}{}{\justifying}{}

\usepackage{xcolor}
\definecolor{emerald}{rgb}{0.31, 0.78, 0.47}

\usepackage{caption}
\usepackage{hyperref}
\hypersetup{
    unicode=true,
    pdfstartview={FitV},
    pdfauthor={Nightwind},
    pdfkeywords={latex, dicas},
    colorlinks=true,
    linkcolor=black,
    citecolor=green,
    filecolor=cyan,
    urlcolor=magenta
}
\usepackage[nameinlink,capitalise]{cleveref}
\usepackage[style=abnt]{biblatex}
\addbibresource{references.bib}
\usepackage{csquotes}

\usepackage{tabularx}
\usepackage{array}
\newcolumntype{M}[1]{>{\raggedleft\arraybackslash}m{#1}}
\newcolumntype{O}[1]{>{\raggedright\arraybackslash}m{#1}}
\newcolumntype{C}[1]{>{\centering\arraybackslash}m{#1}}
\usepackage{multirow}

\usepackage{graphicx}
\usepackage{svg}

\usepackage{multicol}
\usepackage{fontspec}
\setsansfont{Syne}
\setmonofont{Space Mono}
\usefonttheme[onlymath]{serif}

\usepackage{mathtools}
\usepackage{mathrsfs}
\usepackage{xfrac}
\DeclareMathOperator{\sen}{sen}
\newcommand{\intd}[4]{\ensuremath{\int_{#1}^{#2}\left[#3\right]\,\mathsf{d}{#4}}}
\newcommand{\edp}[3][]{\ensuremath{\frac{\partial^{#1}\left[#2\right]}{\partial{#3}^{#1}}}}

\usepackage{listings}
\definecolor{asparagus}{rgb}{0.53, 0.66, 0.42}
\definecolor{lavendergray}{rgb}{0.77, 0.76, 0.82}
\definecolor{pastelmagenta}{rgb}{0.96, 0.6, 0.76}
\definecolor{mistyrose}{rgb}{1.0, 0.89, 0.88}
\definecolor{packagecolor}{rgb}{1.0, 0.0, 0.16}
\definecolor{darkslateblue}{rgb}{0.28, 0.24, 0.55}
\definecolor{forestgreen}{rgb}{0.13, 0.55, 0.13}

\lstdefinestyle{myStyleLatex}{
    language=[LaTeX]TeX,
    backgroundcolor = \color{mistyrose},
    basicstyle = \ttfamily,
    breakatwhitespace = true,
    columns = fullflexible,
    breaklines = true,
    captionpos = a,
    commentstyle = {\footnotesize\color{asparagus}},
    escapeinside = {  {(*}  {*)}  },
    extendedchars = true,
    firstnumber = 1,
    frame = none,
    keepspaces = true,
    keywordstyle = {\bfseries\color{darkslateblue}},
    keywordstyle = [2]{\color{forestgreen}},
    keywordstyle = [3]{\itshape\color{packagecolor}},
    morekeywords = {maketitle,chapter,part, section, subsection, subsubsection, paragraph, subparagraph,familydefault,rmdefault,sfdefault,ttdefault,textsubscript,ttshape,colorbox,textcolor,definecolor,setlength,includegraphics,listoffigures,listoftables,endfirsthead,endhead,endfoot,endlastfoot,arraybackslash,newcolumntype,rowcolor,rowcolors,cellcolor,multirow,setmainfont,documentclass,mathcal,mathbb,mathfrak,mathscr,DeclareMathOperator,cref,autoref,lstdefinestyle,lstset,dfrac,sfrac,lstinline,lstinputlisting},
    morekeywords = [2]{default,arguments},
    morekeywords = [3]{document,command,definition, ambiente,book,report,article,exam, beamer,flushright,flushleft,center,figure,table,tabular,tabularx,longtable,article,book,exam,wrapfigure,equation,split,align,gather,multiline,lstlisting},
    numbers = left,
    numbersep = -1pt,
    numberstyle = \color{lavendergray},
    rulecolor = \color{magenta},
    showspaces = false,
    showstringspaces = false,
    showtabs = false,
    stepnumber = 1,
    stringstyle = \color{pastelmagenta},
    tabsize = 2
}

\usepackage[os=win]{menukeys}

\author{Nightwind}
\institute[CTISM]{Colégio Técnico Industrial de Santa Maria}
\logo{\includegraphics[width=1cm]{../images/photo1.jpg}}
\date{\today}


\title{Inserir Imagens}

\begin{document}

\frame{\titlepage}

\begin{frame}
    \frametitle{Sumário}
    \tableofcontents
\end{frame}

\section{Pacotes}

\begin{frame}[fragile]
    \frametitle{Pacotes}

    \begin{itemize}
        \item Para inserir uma imagem em png e jpg necessita o pacote \texttt{graphicx}. Ele é equivalente ao pacote \texttt{graphics}, mas com mais funções.
        \item O pacote \texttt{wrapfig} serve se o objetivo for colocar uma figura em que o texto continua nos lados.
        \item O pacote \texttt{float}, pode auxiliar à determinar a posição exata da imagem.
        \item O pacote \texttt{svg} pode ser útil se o objetivo for adicionar uma imagem no formato svg.
    \end{itemize}

\end{frame}

\section{Ambiente de imagem}

\begin{frame}[fragile]
    \frametitle{}
    \begin{itemize}
        \item O ambiente da imagem é caracterizado por \lstinline[style=myStyleLatex]!\begin{figure}[<pos>]...\end{figure}!, entre os colchetes, vai a posição em que se deseja inserir a imagem.
        \item A posição pode ser:
              \begin{itemize}
                  \item \texttt{h}: para inserir onde a imagem é citada. Pode ser que a imagem seja adicionada em outro local porque o \LaTeX avalia onde é o melhor local para inserir.
                  \item \texttt{t}: para inserir no início da página.
                  \item \texttt{b}: para inserir no final da página.
                  \item \texttt{p}: para inserir em uma página somente para imagens.
                  \item !: para que o \LaTeX não considere o ``melhor local''
                  \item \texttt{H}: para inserir onde a imagem é citada. Requer pacote \texttt{float}.
              \end{itemize}
    \end{itemize}
\end{frame}

\section{Alinhamento}

\begin{frame}[fragile]
    \frametitle{Alinhamento}

    \begin{itemize}
        \item Por padrão, a imagem é alinhada à esquerda.
        \item Para que seja centralizado, utiliza-se o comando \lstinline[style=myStyleLatex]!\centering!.
        \item Para que seja alinhado à direita \lstinline[style=myStyleLatex]!\raggedright!.
    \end{itemize}

\end{frame}

\section{Legenda e rótulo}
\begin{frame}[fragile]
    \frametitle{Legenda e rótulo}

    \begin{itemize}
        \item A legenda é expressa pelo comando \lstinline[style=myStyleLatex]!\caption[<short caption>]{<long caption>}!.
        \item A parte entre colchetes é opcional. Ela é útil quando a legenda é muito grande, aí utiliza-se um ``resumo'' da legenda para ser expressa no sumário de figuras.
        \item A parte entre chaves é a parte que vai aparecer junto à imagem.
        \item O posicionamento da legenda na imagem varia de acordo com o posicionamento do comando dentro do ambiente de figura.
              \begin{itemize}
                  \item Quando o comando \texttt{caption} vem \textbf{antes} do comando \texttt{includegraphics}, a legenda será em cima da imagem.
                  \item Quando o comando \texttt{caption} vem \textbf{depois} do comando \texttt{includegraphics}, a legenda será embaixo da imagem.
              \end{itemize}
    \end{itemize}

\end{frame}

\begin{frame}[fragile]
    \begin{lstlisting}[style=myStyleLatex]
\begin{figure}[<pos>]
    \centering
    \caption{<text>} % legenda antes
    \label{<text>}
    \includegraphics[<size, angle, ...>]{<image>}
\end{figure}
\end{lstlisting}

\end{frame}

\begin{frame}[fragile]

    \begin{lstlisting}[style=myStyleLatex]
\begin{figure}[<pos>]
    \centering
    \includegraphics[<size, angle, ...>]{<image>}
    \caption{<text>} % legenda depois
    \label{<text>}
\end{figure}
\end{lstlisting}

\end{frame}

\begin{frame}[fragile]
    \frametitle{Referencia cruzada}
    \begin{itemize}
        \item O comando padrão para citar uma label é o \lstinline[style=myStyleLatex]!\ref{<>}!. Recomenda-se utilizar um padrão de nomenclatura dos rótulos, por exemplo, ``fig:teste'' para figuras.
        \item O pacote \texttt{hyperref} oferece um suporte melhor para a citação com o comando \lstinline[style=myStyleLatex]!\autoref{<label>}!. Ele cria um link para a legenda permitindo que, no PDF resultante, pode-se clicar e ir à imagem.
        \item O pacote \texttt{cleveref} também oferece esse suporte. Para citar uma imagem, utiliza-se \lstinline[style=myStyleLatex]!\cref{<label>}!.
        \item \textbf{Para que os dois últimos pacotes funcionem, o comando \texttt{label} precisa aparecer depois de \texttt{caption}.}
    \end{itemize}
\end{frame}

\section{Tamanho da imagem}
\begin{frame}[fragile]
    \frametitle{Tamanho}

    \begin{itemize}
        \item O tamanho é dado nas opções do comando \lstinline[style=myStyleLatex]!\includegraphics[<options>]{<image>}!.
        \item Pode-se determinar \texttt{width} e/ou \texttt{height} proporcionalmente a alguma medida previamente estabelecida ou absolutamente expressada por uma unidade de medida (in, cm, mm); \texttt{scale} em número decimal com ponto; e, o \texttt{angle} em graus sem precisar determinar a unidade.
        \item Exemplo: \lstinline[style=myStyleLatex]!\includegraphics[width={0.7\textwidth}]{<image>}!. Representa que a imagem vai ocupar 70\% da largura atribuida ao texto.
    \end{itemize}

\end{frame}

\section{Inserir svg}
\begin{frame}
    \frametitle{Inserir svg}

    \begin{itemize}
        \item Para inserir imagens no formato svg, basta adicionar o pacote \texttt{svg}.
        \item A única diferença é que no lugar de escrever \texttt{includegraphics}, escreve-se \texttt{includesvg}.
    \end{itemize}

\end{frame}

\section{Figura com texto ao lado}
\begin{frame}[fragile]
    \frametitle{Figura com texto ao lado}

    \begin{itemize}
        \item A estrutura do ambiente é a seguinte:
\begin{lstlisting}[style=myStyleLatex]
\begin{wrapfigure}{<pos box>}{<size box>}
    <alignment>
    \includegraphics[<size image>]{<image>}
\end{wrapfigure}
\end{lstlisting}
        \item Em que a posição do ambiente pode ser \texttt{r} quando for alinhada à direita ou \texttt{l} quando for alinhada à esquerda.
        \item O tamanho do ambiente em que a imagem vai ser inserida, pode ser descrita em unidades de medida ou proporcional.
        \item O alinhamento pode ser \lstinline[style=myStyleLatex]!\centering,\raggedright,\raggedleft!.
        \item O tamanho da imagem pode ser descrita em unidades de medida ou proporcional.
    \end{itemize}

\end{frame}

\section{Lista de Figuras}
\begin{frame}[fragile]
    \frametitle{Lista de Figuras}

    \begin{itemize}
        \item Digitar o comando \lstinline[style=myStyleLatex]!\listoffigures! onde se desejar criar um sumário de figuras.
    \end{itemize}

\end{frame}

\section{Referências}
\begin{frame}[allowframebreaks]
    \frametitle{Referências}
    \nocite{*}
    \printbibliography[keyword={inserirImagem}]
\end{frame}


\end{document}